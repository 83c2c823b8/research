\section{行列因子化の圏}
\begin{defn}\cite{KST06}
	多項式$f$に対して,次数付き行列因子化$M$を次のように定義する:
\[M\coloneq\mf{P_0}{p_0}{p_1}{P_1},\]
$P_0,P_1$は次数付き有限階数自由右$R$-加群,$p_0\colon P_0\rightarrow P_1$は次数$0$の$R$-準同型,$p_1\colon P_1\rightarrow P_0$は次数$2$の$R$-準同型であって,$p_1p_0 = f\cdot\id_{P_0},\ p_0p_1 = f\cdot\id_{P_1}$となるものである.$f$のすべての次数付き行列因子化の集合を$\mathrm{MF}^{\text{gr}}_{R}(f)$と記す.\\
2つの次数付き行列因子化$\ M= \mf{P_0}{p_0}{p_1}{P_1},\ M' =\mf{P_0'}{p_0'}{p_1'}{P_1'}\in\mathrm{MF}^{\text{gr}}_{R}(f)$に対して,その間の射$\Phi = (\phi_0,\phi_1)$を以下のように定める:\\
次数0の$R$-準同型:$\phi_0\colon P_0 \rightarrow P_0'\ \phi_1\colon P_1 \rightarrow P_1' $であって,以下の図式を可換にするもの:
		\[
		\begin{tikzcd}
			P_0\ar[r,"p_0"]\ar[d,"\phi_0"]& P_1\ar[r,"p_1"]\ar[d,"\phi_1"]& P_0\ar[d,"\phi_0"]\\
			P_0'\ar[r,"p_0'",swap]& P_1'\ar[r,"p_1'",swap]& P_0' .
		\end{tikzcd}
	\]
\end{defn}

\begin{defn}\cite{KST06}
	次数付き行列因子化の加法圏$\mathrm{HMF}^{\text{gr}}_{R}(f)$を以下のように定める:
	\[\Ob(\mathrm{HMF}^{\text{gr}}_{R}(f)) \coloneq \mathrm{MF}^{\text{gr}}_{R}(f).\]
	任意の$M,M'\in \mathrm{MF}^{\text{gr}}_{R}(f)$に対して,
	\[\Hom_{\mathrm{HMF}^{\text{gr}}_{R}(f)}(M,M')\coloneq \Hom_{\mathrm{MF}^{\text{gr}}_{R}(f)}(M,M')/\mathbin{\sim}.\]
	ここで,$\Phi,\Phi'\in\Hom_{\mathrm{MF}^{\text{gr}}_{R}(f)}(M,M')$に対して,次数0の$R$-準同型の組:$(h_0,h_1)\colon (P_0\to P_1), (P_1\to P_0)$であって,$\Phi- \Phi' = (\tau^{-h}(p_1')h_0 + h_1p_0,\ p_0'h_1 + \tau^h(h_0)p_1)$となるようなものが存在するとき,同値$\Phi\sim\Phi'$と定める.
		\[
		\begin{tikzcd}
			P_0\ar[r,"p_0"]\ar[d,"\phi_0",swap]& P_1\ar[r,"p_1"]\ar[d,"\phi_1",swap]\ar[ld,"h_1"description,dotted]& P_0\ar[d,"\phi_0"]\ar[ld,"h_0"description,dotted]\\
			P_0'\ar[r,"p_0'",swap]& P_1'\ar[r,"p_1'",swap]& P_0' .
		\end{tikzcd}
	\]
	ただし,$\tau = (1)$は次数シフトである.
\end{defn}

$M= \mf{P_0}{p_0}{p_1}{P_1}\in\mathrm{HMF}^{\text{gr}}_{R}(f)$に対し,
\[P_0 = b_1R\oplus\cdots\oplus b_rR,\quad P_1 = \overline{b_1}R\oplus\cdots \overline{b_r}R.\]
となるような,基底$b_1,\ldots, b_r,\overline{b_1},\ldots, \overline{b_r}$をとると,$2r\times 2r$の行列の組$(Q,S)$で表すことができる.ここで,
\begin{gather*} Q = \begin{pmatrix}
	0 & \phi\\
	\psi & 0
\end{pmatrix},\quad \varphi,\psi \in \Mat_r(R),\\
S = \begin{pmatrix}
	s_1\\
	&\ddots\\
	& & s_r\\
	& & &\overline{s_1}\\
	& & & &\ddots\\
	& & & &&\overline{s_1}\\
\end{pmatrix},\quad s_i = \deg(b_i),\ \overline{s_i} = \deg(\overline{b_i}) - 1.
\end{gather*}
であり,
\[Q^2 = f\cdot I_{2r},\quad -SQ + QS + 2EQ = Q. {\color{red}{why?}}\]

\begin{thm}\cite{KST06}
	$f\in\CC[x,y,z]$をADE型の多項式とし,$\vec{\Delta}$を対応する向きが固定されたディンキン箙とする.このとき,以下の三角圏の同値が存在する:
	\[\mathrm{HMF}^{\text{gr}}_{R}(f)\simeq D^b(\mod \CC\vec{\Delta}).\]
\end{thm}

\begin{defn}\cite{KST06}
	次数付き行列因子化$M = (Q,S)\in\mathrm{HMF}^{\text{gr}}_{R}(f) $に対して,複素数を以下のように対応させる:
	\[\Z(M) \coloneq \Tr(e^{\pi\sqrt{-1}S}).\]
	これをグロタンディーク群上に線型に延長する$\Z\colon K(\mathrm{HMF}^{\text{gr}}_{R}(f))\to\CC$.
\end{defn}

\begin{thm}\cite{KST06}
	$f\in R$をADE型の多項式とする.$\P(\phi)$を$\mathrm{HMF}^{\text{gr}}_{R}(f)$の位相$\phi$の既約対象から生成される充満部分加法圏とする.このとき,$(P(\phi),\Z)$は,$\mathrm{HMF}^{\text{gr}}_{R}(f)$上にBridgeland安定性条件をさだめる.
\end{thm}
