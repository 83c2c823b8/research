\section{安定性条件}
\begin{defn}\cite{Bri07}
$\GL^+_2(\RR)$ の普遍被覆 $\tildeGL^+_2(\RR)$ を次のように定義する.\\
元は$A\in\GL^+_2(\RR)$と$g\colon\RR\to\RR$の組$(A,g)$ であり,次を満たすものである:
\begin{itemize}
  \item $g:\mathbb{R}\to\mathbb{R}$ は単調増加連続函数で $g(\phi+1)=g(\phi)+1$ を満たす.
	\item 次の同一視:
		\[S^1 = \RR/2\ZZ = (\RR^2\backslash \{0\})/\RR_{>0}.\]
		によって,$g$と$A$が$S^1$に誘導する写像が等しい.
\end{itemize}

積は
\[
(A_1,g_1)\cdot(A_2,g_2) = (A_1A_2,\, g_1\ast g_2),
\]
で定義される.ただし, $g_1\ast g_2$ は
\[
(g_1\ast g_2)(\phi) = g_1(g_2(\phi)).
\]
と定める.\\
このとき射影写像 $p:\widetilde{\mathrm{GL}}^+_2(\mathbb{R})\to \mathrm{GL}^+_2(\mathbb{R}),\ (A,g)\mapsto A$ が被覆準同型となり,$(\widetilde{\mathrm{GL}}^+_2(\mathbb{R}),p)$ は $\mathrm{GL}^+_2(\mathbb{R})$ の普遍被覆である.また,$\CC^\ast $は,
		\[
			\begin{array}{ccccc}
				\CC^\ast & \longrightarrow &\GL^+_2(\RR)  \\
				\rotatebox{90}{\in}& &\rotatebox{90}{\in}\\
				x + iy& \longmapsto & \begin{pmatrix}x&-y\\ y&x \end{pmatrix}& .
					\end{array}
\]
によって,$\GL^+_2(\RR)$の部分群なので,$\CC$は$\tildeGL^+_2(\RR)$の部分群であり以下のように埋め込まれる.
		\[
			\begin{array}{ccccc}
				\CC & \longrightarrow &\tildeGL^+_2(\RR)  \\
				\rotatebox{90}{\in}& &\rotatebox{90}{\in}\\
				\lambda& \longmapsto & (e^{i\pi\lambda},\, g\colon x + \Re \lambda)& .
					\end{array}
\]
\end{defn}

\begin{defn}
三角圏 $\mathcal{D}$ 上の安定性条件を $\sigma=(Z,\mathcal{P})$ とする.\\
\bullet \ $(A,g)\in\tildeGL^+_2(\RR)$に対し,右作用 $\sigma\cdot(A,g)=(Z',\mathcal{P}')$ を以下のように定める:
\[
	Z' = A^{-1}\circ Z,\qquad 
\mathcal{P}'(\phi)=\mathcal{P}(g(\phi)),
\]
ただし,$\CC \simeq \RR^2$と同一視して$A^{-1}$の作用を考える.

\bullet \ $\lambda\in \CC$ に対し, $\sigma$への左作用を以下で定める:
\[
\lambda\cdot\sigma := \bigl(e^{-i\pi\lambda}Z,\ \mathcal{P}(\phi+\Re \lambda)\bigr),
\]
\bullet\ 三角圏 $\mathcal{D}$ の自己同値函手 $\Phi\in \Aut(\mathcal{D})$に対して,これが誘導する左作用$\Phi_\ast(\sigma)$を以下で定める:
\[
\Phi_*(\sigma)\ :=\ \bigl(Z\circ \Phi^{-1},\ \Phi(\mathcal{P})\bigr),\quad
\Phi(\mathcal{P})(\phi)\ :=\ \Phi\bigl(\mathcal{P}(\phi)\bigr)\ \ (\phi\in\mathbb{R}).
\]
\end{defn}
ここで,$k\in\ZZ \subset \CC$の左作用と$[k]\in\Aut(\D)$の左作用は,
\begin{equation*}
	k\cdot \sigma = (e^{-ik\pi}Z, \P(\phi + k)) =  (Z\circ [-k], \P(\phi)[k]) = [k]_\ast\cdot \sigma .
\end{equation*}
であるので,同じ作用を定めている.また,$\CC\subset\tildeGL^+_2(\RR)$による右作用と$\CC$の左作用は同じ作用を定めている.

\begin{defn}\cite{Kuz03}
三角圏 $\mathcal{D}$ 上の安定性条件 $\sigma=(Z,\mathcal{P})$ に対して,Serre函手 $S:\mathcal{D}\to\mathcal{D}$ と組 $(A,g)\in \widetilde{\mathrm{GL}}^+_2(\mathbb{R})$ が存在して
\[
S(\sigma)=\sigma \cdot (A,g),
\]
が成り立つとき,$\sigma$ を Serre不変な安定性条件という.

\end{defn}

\begin{defn}
三角圏 $\mathcal{D}$ 上の安定性条件 $\sigma=(Z,\mathcal{P})$ に対して,自己同値 $F:\mathcal{D}\to\mathcal{D}$ と複素数 $s\in\mathbb{C}$ が存在して
\[
	F_*(\sigma)=s\cdot \sigma.
\]
が成り立つとき,$\sigma$ を Gepner型安定性条件という.
\end{defn}

\begin{thm}
	射影直線 \(\mathbb{P}^1\) 上の導来圏 \(D^b(\Coh(\PP^1))\) には Serre不変な安定性条件は存在しない.
\end{thm}
