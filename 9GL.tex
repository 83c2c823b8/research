\section{GL}
\begin{defn}\cite{GL87}
重み付き射影直線(weighted projective line)とは,以下のデータによって定まる射影曲線である:

\begin{itemize}
  \item 正の整数からなる列 $\mathbf{p} = (p_0, p_1, \dots, p_n)$,
  \item $\mathbb{P}^1(k)$ の互いに異なる点の列 $\boldsymbol{\lambda} = (\lambda_0, \lambda_1, \dots, \lambda_n)$,
\end{itemize}

ただし,通常 $\lambda_0 = \infty,\ \lambda_1 = 0,\ \lambda_2 = 1$ と正規化する.次に,アーベル群
\[
L(\mathbf{p}) = \left\langle \vec{x}_0, \vec{x}_1, \dots, \vec{x}_n \ \middle| \ p_0 \vec{x}_0 = p_1 \vec{x}_1 = \cdots = p_n \vec{x}_n \right\rangle .
\]
を定義し,これを次数付き群とする. $\vec{c}\coloneq p_0 \vec{x}_0 = \cdots = p_n \vec{x}_n$ と定める.

$L(\mathbf{p})$ によって次数付けされた多項式環:
\[
	S(\mathbf{p}) = \CC[X_0, X_1, \dots, X_n], \quad \deg X_i = \vec{x}_i. 
\]
において,以下の $L(\mathbf{p})$-斉次イデアル:
\[
I(\mathbf{p}, \boldsymbol{\lambda}) = \left( X_i^{p_i} - X_1^{p_1} + \lambda_i X_0^{p_0} \ \middle|\ i = 2, \dots, n \right).
\]
を用いて商環
\[
	S(\mathbf{p}, \boldsymbol{\lambda}) = k[X_0, X_1, \dots, X_n]\Big{/}I(\mathbf{p}, \boldsymbol{\lambda}) = k[x_0,x_1,\ldots, x_n],\quad \deg(x_i)=\vec{x_i}.
\]
を定義する.
\end{defn}


各次数$\vec{\ell}\in L(\mathbf{p})$は一意的に,
\[\vec{\ell} = \sum_{i=0}^n \ell_i\vec{x_i} + \ell\vec{c}\quad (0\le \ell_i < p_i,\ \ \ell \in \ZZ).\]
とかける.
ここで,
\[\vec{\omega} \coloneq (n-1)\vec{c} - \sum_{i=0}^n\vec{x_i}.\]
と$\vec{\omega}$を定義し,双対化元(dualizing element)と呼ぶ.また,
\[R(\mathbf{p} )= \bigoplus_{\ell=0}^\infty R_{\ell},\quad R_\ell = S_{\ell\vec{c}}.\]
を$S$の核(core)と呼ぶ.

\begin{defn}\cite{GL87}
重み付き射影直線 $\mathbb{X} = \mathbb{C}(\mathbf{p}, \boldsymbol{\lambda})$ に対応する $L(\mathbf{p})$-次数付き環 $S = S(\mathbf{p}, \boldsymbol{\lambda})$ に対して,次のように連接層の圏$\Coh(\mathbb{X})$ を定義する:

\[
\Coh(\mathbb{X}) := \mathrm{mod}^{L(\mathbf{p})}(S)\Big{/}\mathrm{mod}^{L(\mathbf{p})}_0(S).
\]

ここで,
\begin{itemize}
  \item $\mathrm{mod}^{L(\mathbf{p})}(S)$ は $L(\mathbf{p})$-次数付き有限生成 $S$-加群の圏,
	\item $\mathrm{mod}^{L(\mathbf{p})}_0(S)$ は $\mathrm{mod}^{L(\mathbf{p})}(S)$のねじれ加群全体の圏.
\end{itemize}
さらに,$\vec{x} \in L(\mathbf{p})$ に対して,ひねり(twist)とは,$S$-加群 $M$ に対して次のように定義される加群 $M(\vec{x})$ を与える操作である:
\[
M(\vec{x})_{\vec{y}} := M_{\vec{x} + \vec{y}} \quad (\vec{y} \in L(\mathbf{p})).
\]
$(M,\vec{x})\to M(\vec{x})$により,各圏に対して$L(\mathbf{p})$作用が定まる.
\end{defn}

\begin{defn}[局所化]
重み付き射影直線 $\mathbb{X} = \mathbb{C}(\mathbf{p}, \boldsymbol{\lambda})$ に対して,次数付き環 $S = S(\mathbf{p}, \boldsymbol{\lambda})$ とその次数付け群 $L(\mathbf{p})$ を考える.

\end{defn}

\begin{thm}[Serre 双対性]\cite{GL87}
重み付き射影直線 $\mathbb{X} = \mathbb{C}(\mathbf{p}, \boldsymbol{\lambda})$ において,任意の連接層 $\mathcal{F}, \mathcal{G} \in \Coh(\mathbb{X})$ に対して,次の自然なベクトル空間の同型が成り立つ:
\[
\Ext^1(\mathcal{F}, \mathcal{G})^\vee \cong \Hom(\mathcal{G}, \mathcal{F}(\vec{\omega})).
\]
\[
\RHom(\F,\G)^\vee \simeq \RHom(\G,\F(\vec{\omega}))[1]?\]
ここで,
\begin{itemize}
	\item $(-)^\vee := \Hom_{\CC}(-, \CC)$ は $\CC$ 上の線型双対,
  \item $\vec{\omega} = (n - 1)\vec{c} - \sum_{i=0}^n \vec{x}_i$ は dualizing element(双対化元)である.
\end{itemize}
この同型は $\mathcal{F}, \mathcal{G}$ に関して函手的である.
\end{thm}

\begin{lemm}
	重み付き射影直線 $\mathbb{X} = \mathbb{C}(\mathbf{p}, \boldsymbol{\lambda})$において,各直線束$L$は$\Coh(\XX)$における例外対象である.
\end{lemm}
\begin{proof}
	$L = \mathcal{O}(\vec{x})$とかけるので,
	\[\Hom(\mathcal{O}(\vec{x}),\mathcal{O}(\vec{x}))=S_0=\CC.\]
	である.
\end{proof}

\begin{thm}\cite{GL87}
重み付き射影直線 $\mathbb{X} = \mathbb{C}(\mathbf{p}, \boldsymbol{\lambda})$ において,次の加群
\[
T := \bigoplus_{0 \le \vec{x} \le \vec{c}} \mathcal{O}_{\mathbb{X}}(\vec{x}).
\]
をとると,これは $\Coh(\mathbb{X})$ における傾対象である.
\end{thm}


\begin{thm}[Krull--Schmidt 性]\cite{GL87}
重み付き射影直線 $\mathbb{X} = \mathbb{C}(\mathbf{p}, \boldsymbol{\lambda})$ に対して,連接層のアーベル圏
\[
\Coh(\mathbb{X}) := \frac{\mathrm{mod}_+^{L(\mathbf{p})}(S)}{\mathrm{mod}_0^{L(\mathbf{p})}(S)}.
\]
は Krull--Schmidt 圏である.すなわち,任意の対象 $\mathcal{F} \in \Coh(\mathbb{X})$ は直既約対象の有限直和に分解でき,その分解は同型と順序を除いて一意である:
\[
\mathcal{F} \cong \bigoplus_{i=1}^n \mathcal{F}_i \quad (\text{各} \mathcal{F}_i \text{が既約対象}).
\]
\end{thm}

