\section{}
\begin{defn}
$\mathrm{GL}^+_2(\mathbb{R})$ の普遍被覆 $\widetilde{\mathrm{GL}}^+_2(\mathbb{R})$ を次のように定義する.

元は組 $(M,g)$ であり,
\begin{itemize}
  \item $M \in \mathrm{GL}^+_2(\mathbb{R})$,
  \item $g:\mathbb{R}\to\mathbb{R}$ は単調増加連続函数で $g(\phi+1)=g(\phi)+1$ を満たす,
\end{itemize}
ものである.

積は
\[
(M_1,g_1)\cdot(M_2,g_2) = (M_1M_2,\, g_1\ast g_2),
\]
で定義され,ここで $g_1\ast g_2$ は
\[
(g_1\ast g_2)(\phi) = g_1(g_2(\phi)).
\]

このとき射影写像 $p:\widetilde{\mathrm{GL}}^+_2(\mathbb{R})\to \mathrm{GL}^+_2(\mathbb{R}),\ (M,g)\mapsto M$ が被覆準同型となり,$(\widetilde{\mathrm{GL}}^+_2(\mathbb{R}),p)$ は $\mathrm{GL}^+_2(\mathbb{R})$ の普遍被覆である.
\end{defn}

\begin{defn}
三角圏 $\mathcal{D}$ 上の安定性条件 $\sigma=(Z,\mathcal{P})$ に対して,セール函手 $S:\mathcal{D}\to\mathcal{D}$ と組 $(M,g)\in \widetilde{\mathrm{GL}}^+_2(\mathbb{R})$ が存在して
\[
S(\sigma)=\sigma \cdot (M,g),
\]
が成り立つとき,$\sigma$ を \textbf{Serre不変な安定性条件} という.

ここで右作用 $(M,g)\cdot\sigma=(Z',\mathcal{P}')$ は
\[
	Z' = M^{-1}\circ Z,\qquad 
\mathcal{P}'(\phi)=\mathcal{P}(g(\phi)).
\]
で定義される.
\end{defn}

\begin{defn}
三角圏 $\mathcal{D}$ 上の安定性条件を $\sigma=(Z,\mathcal{P})$ とする.
任意の $\lambda\in \CC$ に対し
\[
\lambda\cdot\sigma := \bigl(e^{-\pi i \lambda}Z,\ \mathcal{P}(\phi+\Re(\lambda))\bigr)
\]
で定まる安定性条件によって, $\sigma$ の $\CC$ 作用を定める.
三角圏 $\mathcal{D}$ の自己同値函手 $\Phi\in \mathrm{Aut}(\mathcal{D})$ に対して,
\[
\Phi_*(\sigma)\ :=\ \bigl(Z\circ \Phi^{-1},\ \Phi(\mathcal{P})\bigr),\quad
\Phi(\mathcal{P})(\phi)\ :=\ \Phi\bigl(\mathcal{P}(\phi)\bigr)\ \ (\phi\in\mathbb{R})
\]
で定まる安定性条件によって,$\sigma$の $\Phi$ による作用を定める.
\end{defn}

\begin{defn}
三角圏 $\mathcal{D}$ 上の安定性条件 $\sigma=(Z,\mathcal{P})$ に対して,自己同値 $F:\mathcal{D}\to\mathcal{D}$ と複素数 $s\in\mathbb{C}$ が存在して
\[
	F_*(\sigma)=s\cdot \sigma.
\]
が成り立つとき,$\sigma$ を \textbf{Gepner型安定性条件} という.
\end{defn}


\begin{thm}
	射影直線 \(\mathbb{P}^1\) 上の導来圏 \(D^b(\Coh(\PP^1))\) には Serre不変な安定性条件は存在しない.
\end{thm}
