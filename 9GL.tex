\section{安定性条件}
\begin{defn}\cite{Bri07}
$\GL^+_2(\RR)$ の普遍被覆 $\tildeGL^+_2(\RR)$ を次のように定義する.\\
元は$A\in\GL^+_2(\RR)$と$g\colon\RR\to\RR$の組$(A,g)$ であり,次を満たすものである:
\begin{itemize}
  \item $g:\mathbb{R}\to\mathbb{R}$ は単調増加連続函数で $g(\phi+1)=g(\phi)+1$ を満たす.
	\item 次の同一視:
		\[S^1 = \RR/2\ZZ = (\RR^2\backslash \{0\})/\RR_{>0}.\]
		によって,$g$と$A$が$S^1$に誘導する写像が等しい.
\end{itemize}

積は
\[
(A_1,g_1)\cdot(A_2,g_2) = (A_1A_2,\, g_1\ast g_2),
\]
で定義される.ただし, $g_1\ast g_2$ は
\[
(g_1\ast g_2)(\phi) = g_1(g_2(\phi)).
\]
と定める.\\
このとき射影写像 $p:\widetilde{\mathrm{GL}}^+_2(\mathbb{R})\to \mathrm{GL}^+_2(\mathbb{R}),\ (A,g)\mapsto A$ が被覆準同型となり,$(\widetilde{\mathrm{GL}}^+_2(\mathbb{R}),p)$ は $\mathrm{GL}^+_2(\mathbb{R})$ の普遍被覆である.また,$\CC^\ast $は,
		\[
			\begin{array}{ccccc}
				\CC^\ast & \longrightarrow &\GL^+_2(\RR)  \\
				\rotatebox{90}{\in}& &\rotatebox{90}{\in}\\
				x + iy& \longmapsto & \begin{pmatrix}x&-y\\ y&x \end{pmatrix}& .
					\end{array}
\]
によって,$\GL^+_2(\RR)$の部分群なので,$\CC$は$\tildeGL^+_2(\RR)$の部分群であり以下のように埋め込まれる.
		\[
			\begin{array}{ccccc}
				\CC & \longrightarrow &\tildeGL^+_2(\RR)  \\
				\rotatebox{90}{\in}& &\rotatebox{90}{\in}\\
				\lambda& \longmapsto & (e^{i\pi\lambda},\, g\colon x + \Re \lambda)& .
					\end{array}
\]
\end{defn}

\begin{defn}
三角圏 $\mathcal{D}$ 上の安定性条件を $\sigma=(Z,\mathcal{P})$ とする.\\
\bullet \ $(A,g)\in\tildeGL^+_2(\RR)$に対し,右作用 $\sigma\cdot(A,g)=(Z',\mathcal{P}')$ を以下のように定める:
\[
	Z' = A^{-1}\circ Z,\qquad 
\mathcal{P}'(\phi)=\mathcal{P}(g(\phi)),
\]
ただし,$\CC \simeq \RR^2$と同一視して$A^{-1}$の作用を考える.この作用によって対象同士の位相の大小関係は不変であるため,半安定対象は変わらない.

\bullet \ $\lambda\in \CC$ に対し, $\sigma$への左作用を以下で定める:
\[
\lambda\cdot\sigma := \bigl(e^{-i\pi\lambda}Z,\ \mathcal{P}(\phi+\Re \lambda)\bigr),
\]
\bullet\ 三角圏 $\mathcal{D}$ の自己同値函手 $\Phi\in \Aut(\mathcal{D})$に対して,これが誘導する左作用$\Phi_\ast(\sigma)$を以下で定める:
\[
\Phi_*(\sigma)\ :=\ \bigl(Z\circ \Phi^{-1},\ \Phi(\mathcal{P})\bigr),\quad
\Phi(\mathcal{P})(\phi)\ :=\ \Phi\bigl(\mathcal{P}(\phi)\bigr)\ \ (\phi\in\mathbb{R}).
\]
\end{defn}
ここで,$k\in\ZZ \subset \CC$の左作用と$[k]\in\Aut(\D)$の左作用は,
\begin{equation*}
	k\cdot \sigma = (e^{-ik\pi}Z, \P(\phi + k)) =  (Z\circ [-k], \P(\phi)[k]) = [k]_\ast\cdot \sigma .
\end{equation*}
であるので,同じ作用を定めている.また,$\CC\subset\tildeGL^+_2(\RR)$による右作用と$\CC$の左作用は同じ作用を定めている.

\begin{defn}\cite{Kuz03}
	三角圏 $\mathcal{D}$ 上の安定性条件 $\sigma=(Z,\mathcal{P})$ に対して,Serre函手 $\mathcal{S}:\mathcal{D}\to\mathcal{D}$ と組 $(A,g)\in \widetilde{\mathrm{GL}}^+_2(\mathbb{R})$ が存在して
\[
	\mathcal{S}(\sigma)=\sigma \cdot (A,g),
\]
が成り立つとき,$\sigma$ を Serre不変な安定性条件という.
\end{defn}

\begin{defn}\cite{Tod13}
三角圏 $\mathcal{D}$ 上の安定性条件 $\sigma=(Z,\mathcal{P})$ に対して,自己同値 $F:\mathcal{D}\to\mathcal{D}$ と複素数 $\xi\in\mathbb{C}$ が存在して
\[
	F_*(\sigma)=\xi\cdot \sigma.
\]
が成り立つとき,$\sigma$ を$(F,\xi)$に関するGepner型安定性条件という.
\end{defn}

\begin{rem}
	$\sigma$が$(F,\xi)$に関するGepner型安定性条件であるとき,任意の$s\in
	\CC$に対して,$s \cdot \sigma$も$(F,\xi)$に関してGepner型安定性条件である.
\end{rem}

\begin{thm}
	重み付き射影直線 \(\PP_{A,\Lambda}\) 上の導来圏 \(D^b(\Coh\PP_{A,\Lambda})\) 上のSerre函手を$\mathcal{S}$とする.このとき,以下は同値である:\vspace{-3mm}
	\begin{itemize}
		\item $(\mathcal{S},\xi)$に関して$\sigma$がGepner型であるような安定性条件$\sigma$が存在する.
		\item オービフォールドオイラー標数$\chi_A$が$0$である.
	\end{itemize}
\end{thm}
\begin{proof}
		\[
			\begin{array}{ccccc}
				(\rk,\deg)&\colon K(\PP_{A,\Lambda})&\longrightarrow &\ZZ^{\oplus 2}\\
									&\rotatebox{90}{\in}& &\rotatebox{90}{\in}\\
													 &X & \longmapsto & (\rk(X),\deg(X))& .
					\end{array}
\]
と定めると,$K(\PP_{A,\Lambda})$.Serre函手の擬逆函手$\mathcal{S}^{-1}$が$\ZZ^{\oplus 2}$に誘導する線型写像$[\mathcal{S}^{-1}]$は,
\begin{align*}
	\begin{pmatrix}
		-1 & 0\\ -a \cdot \chi_{A}& -1
	\end{pmatrix},
\end{align*}
と行列表示される.自己同値函手がSerre函手であるようなGepner型安定性条件が存在すると仮定すると,
\begin{align*}
	e^{-i\xi\pi} Z =  
	\begin{pmatrix}
		-1 & 0\\ -a \cdot \chi_{A}& -1
	\end{pmatrix}
	Z
\end{align*}
したがって,$e^{-is\pi}$は$[\mathcal{S}^{-1}]$の固有値になっていることがわかり,$[\mathcal{S}^{-1}]$の固有値は$-1$のみであるため,$e^{-is\pi}=-1$がしたがう.
\begin{align*}
	e^{-i\xi\pi}Z  =
	\begin{pmatrix}
		-1 & 0\\ -a \cdot \chi_{A}& -1
	\end{pmatrix} Z	
	\intertext{$Z$は正則なので,}	
	\begin{pmatrix}
		e^{-i\xi\pi} & 0\\ 0 & e^{-i\xi\pi} 
	\end{pmatrix} = 	
	\begin{pmatrix}
		-1 & 0\\ -a \cdot \chi_{A}& -1
	\end{pmatrix} 	
\end{align*}
等しくなる必要条件が$\chi_A = 0$であることがわかる.
\end{proof}

\begin{thm}\cite{ESP09}
	$G$を$X$に作用する有限群とし,$\Gamma_X\subset\Stab(D^b(X))$を$G$-不変な安定性条件とする.このとき,$\Gamma_X$は,$\Stab(D^b(X))$の閉部分多様体になり,忘却函手$\Forg_G$は以下の条件を満たす閉埋め込みを誘導する:
	\[\Forg_G^{-1}\colon \Gamma_X \hookrightarrow \Stab(D^b_G(X)).\]
	\bullet\ $\Forg_G^{-1}(\sigma)$に関して半安定であるものと$D^b_G(X)$の対象であって$\sigma$に関して半安定であるもが一致する.
\end{thm}

\begin{lemm}
	$\A$をアーベル圏とする.零対象でない単純対象:$S_1,S_2\in\A$に対して非零射$f\colon S_1\to S_2$が存在すれば,$f$は同型射である.
\end{lemm}
\begin{proof}
		$f$を非零射とすると$S_1$が単純対象であることから$\Ker f = 0$.$0\subsetneq\Im f \subseteq S_2$で$S_2$が単純対象であることから$\Im f\simeq S_2$がわかり,同型である.
\end{proof}

\begin{thm}\cite{Bri07}
	$C$を楕円曲線とする.このとき,$\widetilde{\mathrm{GL}}^+_2(\mathbb{R})$の$\Stab(C)$への作用は自由・推移的である.したがって,位相同型:
	\[\Stab(C)\cong \widetilde{\mathrm{GL}}^+_2(\mathbb{R}).\]
が成り立つ.
\end{thm}
\begin{proof}
	\bullet \ 任意の直既約な対象$0\neq X\in\Coh C$は,任意の$\sigma\in\Stab(C)$に対し半安定である.\\
$X$が半安定でないと仮定するとHN分解を考えることで,$0$でない$Y_1,Y_2\in D^b(\Coh C)$で
\[Y_1\longrightarrow X\longrightarrow Y_2 \longrightarrow Y_1[1].\]
となるものが存在する.Serre双対性より,
\[\Hom_{D^b(\Coh C)}(Y_2,Y_1[1])\simeq \Hom_{D^b(\Coh C)}(Y_1,Y_2)^\vee = 0.\]
したがって,$X\simeq Y_1\oplus Y_2$となり,$X$が直既約であることに矛盾.\\\\
\bullet \ $\sigma = (Z,\P)\in\Stab(C)$に対して,$Z\colon \Gamma\to\CC$は$Z_{\RR}\colon\Gamma_{\RR}\cong \RR^2 $を誘導する. \\
$Z_{\RR}$が同型でないとすると,$Z_{\RR}$の像を含む$\ell\subset \RR^2$が存在する.したがって,ある$\phi\in (0,1]$が存在して,
\[\A = \P((0,1]) = \P(\phi).\]
となる.したがって,$\A$は単純対象で生成される.ここで,$X_1,X_2\in\A$を同型でない単純対象とするとSerre双対定理も用いて,
\[\Hom_{D^b(\Coh C)}(X_1,X_2) = \Hom_{D^b(\Coh C)}(X_2,X_1) \simeq \Hom_{D^b(\Coh C)}(X_1,X_2[1]) = 0.\]
がわかる.したがって,$\chi(X_1,X_2) = 0$である.$v(X_1) \coloneq (r_1,d_1),\, v(X_2) \coloneq (r_2,d_2)$と定めるとHirzebruch-Riemann-Roch の定理より,
\[r_1d_2 - r_2d_1 = 0.\]
$v(X_1)$と$v(X_2)$が$\Gamma_\RR$で比例することがわかり,$\A$はこれらで生成されていたので任意の対象$X_1,X_2\in D^b(C)$に対して,$v(X_1)$と$v(X_2)$が比例する.しかし,$v\colon K(C)\to \ZZ^{\oplus 2}$は全射であったので矛盾.\\\\
\bullet\ 任意の$\sigma = (Z,\P)\in\Stab(C)$に対して,$Z_{\RR}\colon \RR^2 \to \RR^2$は向きを保つ.\\
$\rk(X)=\rk(Y)=1,\, \deg(X)<\deg(Y)$となるように,$X,Y\in\Coh C$をとる.$X,Y$は直既約対象なので半安定であり,$\phi,\psi\in\RR$が存在して,$A\in\P(\phi),\, B\in\P(\psi)$となる.
\[\Hom_{D^b(\Coh C)}(A,B)\simeq \Hom_{D^b(\Coh C)}(B,A[1])\neq 0.\]
したがって,$\phi\leq \psi \leq \phi + 1$.向きを保つ.\\\\
\bullet\ 以下の写像:
		\[
			\begin{array}{ccccc}
				\sigma_0\cdot &\colon \widetilde{\mathrm{GL}}^+_2(\mathbb{R}) &\longrightarrow &\Stab(C)\\
									&\rotatebox{90}{\in}& &\rotatebox{90}{\in}\\
													 &(g,A) & \longmapsto & \sigma_0\cdot (g,A)& .
					\end{array}
\]
は全単射である.\\
まず,単射性を示す.
\[\sigma_0\cdot (g,A) = \sigma_0.\]
のとき,このとき,$A^{-1}\circ Z_0 = Z_0$であるが$Z_0\colon \Gamma_{\RR}\to\RR^2$が同型なので,$A = I$となる.このとき,$f(x) = x + 2m\, (m\in\ZZ)$という函数になるが$\Coh(C)[2m] = \Coh(C)$より$m = 0$である.したがって,$(g,A) = (\id_{\RR},I)$が示され単射性がわかった.次に,全射性を示す.{\color{red}{要証明:全射性}}
\end{proof}

\begin{thm}
	楕円曲線$C$に対して,
	\[\Aut(D^b(\Coh C)) \simeq \left(\Aut(C)\ltimes\Pic^0(C)\right)\ltimes\ZZ[1]\ltimes\SL(2,\ZZ).\]
\end{thm}

\begin{prop}
	$C$を楕円曲線とすると,$\Stab(C)$は任意の有限群の作用で不変である.したがって,任意の群$G$に対して$\Gamma_X = \Stab(C)$となり,
	\[\Forg_G^{-1}\colon  \Stab(D^b(\Coh C))\hookrightarrow \Stab(D^b_{G}(\Coh C)) \simeq \Stab(D^b(\Coh \left[X/G\right])).\]
と埋め込むことができる.
\end{prop}
