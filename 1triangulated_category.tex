\section{三角圏の定義}
\begin{defn}\cite[p.170]{KS06}
	以下の条件をみたす圏$\C$を前加法圏(preadditive category)という.
	\vspace{-3mm}
	\begin{itemize}
		\item[(i)]
			任意の$X,Y\in\C$に対して,射の集合$\Hom_{\C}(X,Y)$が加法群になる.
		\item[(ii)]
			任意の$X,Y,Z\in\C$に対して,合成写像
		\[
			\begin{array}{ccccc}
				\circ\colon\Hom_{\C}(Y,Z) \times \Hom_{\C}(X,Y) & \longrightarrow & \Hom_{\C}(X,Z) \\
				\rotatebox{90}{\in}& &\rotatebox{90}{\in}\\
				(g, f) & \longmapsto & g \circ f& .
					\end{array}
\]
が双線型である.つまり,任意の$g,g'\in\Hom_{\C}(Y,Z),\ f,f'\in\Hom_{\C}(X,Y)$に対して,
\begin{gather*}
	(g+g')\circ f = g\circ f + g'\circ f\\
	g\circ(f+f') = g\circ f + g\circ f'.
\end{gather*}
が成り立つ.
	\end{itemize}
\end{defn}

\begin{defn}\cite[p.171]{KS06}
	$\C$を前加法圏とする.このとき,対象$X\oplus Y\in\C$と$\iota_{X}\in\Hom_{\C}(X,X\oplus Y),\iota_{Y}\in\Hom_{\C}(Y,X\oplus Y)$の三つ組$(X\oplus Y,\iota_X,\iota_Y)$であって,以下の普遍性を満たすものを$X,Y\in\C$の直和(direct sum)という.
	\[\begin{tikzcd}
		X\ar[r,"\iota_X"]\ar[rd,"f_X",swap]& X\oplus Y\ar[d,dotted,"\exists !h",swap] &Y\ar[l,"\iota_Y",swap]\ar[ld,"f_Y"]\\
																			 &Z&.
\end{tikzcd}\]
任意の$Z\in\C$と$f_X\in\Hom_{\C}(X,Z),\ f_Y\in\Hom_{\C}(Y,Z)$に対して,$h\circ \iota_X=f_X,\ h\circ\iota_Y=f_Y$となる$h\in\Hom_{\C}(X\oplus Y,Z)$が一意に存在する.
\end{defn}

\begin{defn}\cite[p.171]{KS06}
	以下の条件をみたす前加法圏$\C$を加法圏(additive category)という.
	\vspace{-3mm}
	\begin{itemize}
	\item[(i)]零対象(始対象かつ終対象)である$0\in\C$をを持つ.
	\item[(ii)]任意の対象$X,Y\in\C$に対し直和$X\oplus Y$が存在する.
	\end{itemize}
	\vspace{-3mm}
\end{defn}

\begin{defn}\cite[p.171]{KS06}
	 	加法圏$\C,\D$に対し,函手$F\colon\C\to\D$が以下の条件を満たすとき加法函手(additve functor)という.
		\begin{itemize}
			\item[(1)]
				任意の$X,Y\in\C$に対し,
				\[
			\begin{array}{ccccc}
				F\colon\Hom_{\C}(X,Y) & \longrightarrow & \Hom_{\D}(F(X),F(Y)) \\
				\rotatebox{90}{\in}& &\rotatebox{90}{\in}\\
				f & \longmapsto & F(f)&.
					\end{array}
				\]
				が加法群の準同型写像になっている.つまり任意の$f,g \in\Hom_{\C}(X,Y)$に対し,
				\[F(f + g) = F(f) + F(g).\]
				となる.
			\item[(2)]
				$0\in\C$に対し,$F(0)=0$
			\item[(3)]
			直和を保つ.つまり,任意の対象$X,Y\in\C$に対して以下が成り立つ.
				\[F(X\oplus Y)\simeq F(X)\oplus F(Y).\]
		\end{itemize}
\end{defn}

\begin{defn}\cite[p.175]{KS06}
	$\C$を加法圏とする.このとき,$f\in\Hom_{\C}(X,Y)$の核(kernel)とは$\Ker f\in\C$と$\ker f\in\Hom_{\C}(\Ker f,X)$の組$(\Ker f,\ker f)$であって,以下の普遍性をみたすものである.
	\[\begin{tikzcd}
		K\ar[rd,"k"]\ar[d,"\exists !h",dotted,swap]\\
		\Ker f\ar[r,"\ker f",swap]&X\ar[r,"f",swap]&Y.
\end{tikzcd}\]
\begin{itemize}
	\item[(i)]
		$f\circ\ker f=0$
	\item[(ii)]
		任意の$K\in\C$と$k\in\Hom_{\C}(K,X)$で$f\circ k=0$を満たすものに対して,一意に$h\in\Hom_{\C}(K,\Ker f)$が存在して,$\ker f\circ h=k$となる.
\end{itemize}
$f\in\Hom_{\C}(X,Y)$の$\C$の反対圏での核を余核(cokernel)とよび,$(\Cok f,\cok f)$と記す.\\
また,$f$の像(Image) $\Im f$,余像(coimage) $\Coim f$を以下のように定義する.
\[\Im f\coloneq \Ker(\cok f)\quad \Coim f\coloneq \Cok(\ker f).\]
このとき,普遍性により以下の図式を可換にする射$\Coim f\to \Im f$が一意的に存在することがわかる.
	\[\begin{tikzcd}
		\Ker f\ar[r,"\ker f"]& X \ar[r,"f"]\ar[d,"p",swap]& Y\ar[r, "\cok f"]& \Cok f\\
												 &\Coim f\ar[ru,dotted]\ar[r,dotted] & \Im f\ar[u,"i",swap].
\end{tikzcd}\]
ただし,$p=\ker(\cok f),\, i=\cok(\ker f)$とした.
\end{defn}

\begin{defn}\cite[p.175]{KS06}
	加法圏$\A$が以下を満たすとき,アーベル圏(abelian category)という:
	\vspace{-3mm}
	\begin{itemize}
		\item[(i)]
			$\A$の任意の射$f$に対し,核$\Ker f$と余核$\Cok f$が存在する.
		\item[(ii)]
			$\A$の任意の射$f$に対し,自然な準同型\ $\Coim(f)\simeq\Im f$が同型となる.
	\end{itemize}
\end{defn}
	加法圏$\D$と自己同値函手$[1]\colon\D\to\D$与えられているとき,$\D$における三角形とは以下の形の射の列をいう.
	\[X\xrightarrow{f}Y\xrightarrow{g} Z\xrightarrow{h} X[1].\]

	\begin{defn}\cite[p.243]{KS06}
	加法圏$\D$に自己同値函手$[1]$と完全三角形(distinguished triangle)と呼ばれる三角形の集合が与えられていて,これらが以下の条件を満たすとき,$\D$を三角圏(triangulated category)と呼ぶ.三角形が完全三角形であるとき,「三角形が完全である.」とも表現し,以下の記述も用いることにする.
	\[X\xrightarrow{f}Y\xrightarrow{g} Z\xrightarrow{h} X[1]\hspace{3mm}(完全). \]
	\vspace{-3mm}
	\begin{itemize}
		\item[(TR1)]
			任意の$X\in\D$に対して,
			\[X\xrightarrow{\id_X}X\rightarrow 0 \rightarrow X[1].\]
			は完全三角形である.
		\item[(TR2)]
		以下の2つの三角形とその間の射$f,g,h$が
			\[
		\begin{tikzcd}
			X_1\ar[r]\ar[d,"f"]& X_2\ar[r]\ar[d,"g"]& X_3\ar[r]\ar[d,"h"] & X_1[1]\ar[d,"f\texttt{[1]}"]\\
			Y_1\ar[r]& Y_2\ar[r]& Y_3\ar[r] & Y_1[1].\\
		\end{tikzcd}
			\]
		このとき,$f,g,h$が同型で$X_1\rightarrow X_2\rightarrow X_3 \rightarrow X_1[1]$が完全三角形なら$Y_1\rightarrow Y_2\rightarrow Y_3 \rightarrow Y_1[1]$も完全三角形である.
		\item[(TR3)]
			任意の射$X\xrightarrow{f}Y$は,$X\xrightarrow{f} Y\rightarrow Z \rightarrow X[1]$と完全三角形に拡張できる.
	\item[(TR4)]
		三角形
		\[X_1\xrightarrow{u} X_2\xrightarrow{v} X_3\xrightarrow{w}  X_1[1].\]
		が完全三角形であることと
		\[X_2\xrightarrow{v} X_3\xrightarrow{w} X_1[1]\xrightarrow{-u[1]}  X_2[1].\]
		が完全三角形であることが同値である.

	\item[(TR5)]
以下の2つの完全三角形と図式を可換にする$f,g$が存在したとする.
		\[
		\begin{tikzcd}
			X_1\ar[r]\ar[d,"f"]& X_2\ar[r]\ar[d,"g"]& X_3\ar[r]\ar[d,"h",dotted] & X_1[1]\ar[d,"f\texttt{[1]}"]\\
			Y_1\ar[r]& Y_2\ar[r]& Y_3\ar[r] & Y_1[1].\\
		\end{tikzcd}
	\]
	このとき,すべての四角形を可換にする$h$が存在する.

	\item[(TR6)]
		(八面体公理)3つの完全三角形
			\[
				\begin{tikzcd}[row sep=5pt]
			X \ar[r,"f"]& Y\ar[r,"h"]& Z' \ar[r]& X[1]\\
			X \ar[r,"g\circ f"]& Z\ar[r,"l"]& Y' \ar[r]& X[1]\\
			Y \ar[r,"g"]& Z\ar[r,"k"]& X' \ar[r]& Y[1].\\
		\end{tikzcd}
			\]
			に対して,以下の図式のすべての四角形を可換にし,4行目を完全三角形にするような$u\in\Hom_{\D}(Z',Y'),v\in\Hom_{\D}(Y',X'),w\in\Hom_{\D}(X',Z'[1])$が存在する.
			\[
		\begin{tikzcd}
			X \ar[r,"f"]\ar[d,equal]& Y\ar[r,"h"]\ar[d,"g",swap]& Z'\ar[d,"u",dotted] \ar[r]& X[1]\ar[d,equal]\\
			X \ar[r,"g\circ f"]\ar[d,"f",swap]& Z\ar[r,"l"]\ar[d,equal]& Y'\ar[d,"v",dotted] \ar[r]& X[1]\ar[d,"f\texttt{[1]}"]\\
			Y \ar[r,"g"]\ar[d,"h",swap]& Z\ar[r,"k"]\ar[d,"l",swap]& X' \ar[r]\ar[d,equal]& Y[1]\ar[d,"h\texttt{[1]}"]\\
			Z' \ar[r,"u",dotted]& Y'\ar[r,"v",dotted]& X' \ar[r,"w",dotted]& Z'[1].\\
		\end{tikzcd}
			\]
	\end{itemize}
\end{defn}

\begin{defn}\cite[p.243]{KS06}
三角圏 $\D$ において,2つの完全三角形
\[
X \xrightarrow{f} Y \xrightarrow{g} Z \xrightarrow{h} X[1], \quad
X' \xrightarrow{f'} Y' \xrightarrow{g'} Z' \xrightarrow{h'} X'[1].
\]
が与えられているとする.

このとき,射の三つ組 $(\alpha, \beta, \gamma)$(ただし $\alpha \colon X \to X'$,$\beta \colon Y \to Y'$,$\gamma \colon Z \to Z'$)が以下の図式が可換にするとき,完全三角形の間の射(morphism of triangles)であるという:
\[
\begin{tikzcd}
X \arrow{r}{f} \arrow{d}{\alpha} & Y \arrow{r}{g} \arrow{d}{\beta} & Z \arrow{r}{h} \arrow{d}{\gamma} & X[1] \arrow{d}{\alpha[1]} \\
X' \arrow{r}{f'} & Y' \arrow{r}{g'} & Z' \arrow{r}{h'} & X'[1].
\end{tikzcd}
\]
すなわち,以下が成り立つ:
\[
\beta \circ f = f' \circ \alpha, \quad
\gamma \circ g = g' \circ \beta, \quad
\alpha[1] \circ h = h' \circ \gamma.
\]
\end{defn}
