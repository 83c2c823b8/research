\section{三角圏の基本的な性質}
\begin{prop}\cite[p.245]{KS06}\label{gf=0}
三角圏 $\D$ において,完全三角形
\[
X \xrightarrow{f} Y \xrightarrow{g} Z \xrightarrow{h} X[1].
\]
が与えられたとき,合成 $g \circ f = 0$ が成り立つ.
\end{prop}

\begin{proof}
三角圏の公理 (TR1) と(TR5)により,以下の図式
		\[
\begin{tikzcd}
	X\ar[r,"\id"]\ar[d,"\id"]&X \ar[r]\ar[d,"f"]& 0\ar[r]\ar[d]& X[1]\ar[d,"f\texttt{[1]}"]\\
	X\ar[r,"f"]&Y \ar[r,"g"]& Z\ar[r]& X[1].
\end{tikzcd}
	\]
	が可換になることから$g\circ f = 0$である.
\end{proof}

\begin{prop}\cite[p.245]{KS06}
三角圏 $\D$ において,対象 $X \in \D$ と同型射 $f \colon X \xrightarrow{\sim} Y$ が与えられたとする.
このとき,三角
\[
X \xrightarrow{f} Y \to 0 \to X[1].
\]
は完全三角形である.
\end{prop}

\begin{proof}
三角圏の公理 \textnormal{(TR1)} により,任意の対象 $X$ に対して恒等射による完全三角形と以下の完全三角形の同型
		\[
\begin{tikzcd}
	X\ar[r,"f"]\ar[d,equal]&Y \ar[r]\ar[d,"f\inv"]& 0\ar[r]\ar[d]& X[1]\ar[d,equal]\\
	X\ar[r,"\id"]&X \ar[r,"g"]& 0\ar[r]& X[1].
\end{tikzcd}
	\]
	が存在するので,上の行が完全三角形であることがわかる.
\end{proof}

\begin{defn}\cite[p.245]{KS06}
	三角圏$\D$からアーベル圏$\A$への加法函手$H\colon \D\to\A$がコホモロジー的であるとは,$\D$における任意の完全三角形
	\[X\xrightarrow{f}Y\xrightarrow{g}Z\xrightarrow{h}X[1].\]
	に対して,$\A$において
	\[H(X)\xrightarrow{H(f)}H(Y)\xrightarrow{H(g)}H(Z).\]
	が完全列になることである.
\end{defn}

\begin{prop}\cite[p.245]{KS06}
	$\D$を三角圏,$W\in \D$としたとき,函手
	\begin{gather*}
		\Hom_{\D}(W,-)\colon \D\to\Mod\ZZ \\
		\Hom_{\D}(-,W)\colon \D^{\mathrm{op}}\to\Mod\ZZ\ .
	\end{gather*}
	はコホモロジー的函手である.
\end{prop}
\begin{proof}
	$X\rightarrow Y \rightarrow Z\rightarrow X[1]$を完全三角形,$W\in\D$とする.このとき,以下の完全列
	\[\Hom_{\D}(W,X)\xrightarrow{\Hom_{\D}(W,f)}\Hom_{\D}(W,Y)\xrightarrow{\Hom_{\D}(W,g)}\Hom_{\D}(W,Z).\]
	が完全であることを示す.命題\ref{gf=0}より,$\Im(\Hom_{\D}(W,f))\subset \Ker(\Hom_{\D}(W,g))$がわかる.
	逆の包含に関してはTR4,TR5より,$g\circ\varphi = 0$をみたす$\varphi\colon W\to Y$を任意にとると,以下の図式を可換にする$\psi\colon W\to X$が存在する.
		\[
\begin{tikzcd}
	W\ar[r,"\id"]\ar[d,dotted,"\psi"]&W \ar[r]\ar[d,"\varphi"]& 0\ar[r]\ar[d]& W[1]\ar[d,"\psi\texttt{[1]}"]\\
	X\ar[r,"f"]&Y \ar[r,"g"]& Z\ar[r]& X[1].
\end{tikzcd}
	\]
つまり,$\varphi = f\circ\psi$である.したがって,$\Im(\Hom_{\D}(W,f))= \Ker(\Hom_{\D}(W,g))$が示された.
\end{proof}



\begin{prop}\cite[p.246]{KS06}
三角圏 $\D$ において,次の2つの完全三角形
\[
X \xrightarrow{f} Y \xrightarrow{g} Z \xrightarrow{h} X[1], \quad
X' \xrightarrow{f'} Y' \xrightarrow{g'} Z' \xrightarrow{h'} X'[1].
\]
とその間の射,
\[
\begin{tikzcd}
X \arrow{r}{f} \arrow{d}{\alpha}[swap]{\sim} & Y \arrow{r}{g} \arrow{d}{\beta}[swap]{\sim} & Z \arrow{r}{h} \arrow[dotted]{d}{\gamma} & X[1] \arrow{d}{\alpha[1]} \\
X' \arrow{r}{f'} & Y' \arrow{r}{g'} & Z' \arrow{r}{h'} & X'[1].
\end{tikzcd}
\]
が与えられており,$\alpha, \beta$ が同型であれば,$\gamma$ も同型となる.
\end{prop}
\begin{proof}
	$W\in\D$を任意にとり,$\Hom_{\D}(W,-)$を作用させるとコホモロジー的函手であることにより,以下の完全列が存在する.
\[\begin{tikzcd}
	\Hom_{\D}(W,X)\ar[r]\ar[d,"\rotatebox{90}{\sim}"]&\Hom_{\D}(W,Y)\ar[r]\ar[d,"\rotatebox{90}{\sim}"]&\Hom_{\D}(W,Z)\ar[r]\ar[d]&\Hom_{\D}(W,X[1])\ar[r]\ar[d,"\rotatebox{90}{\sim}"]&\Hom_{\D}(W,Y[1])\ar[d,"\rotatebox{90}{\sim}"]\\
	\Hom_{\D}(W,X')\ar[r]&\Hom_{\D}(W,Y')\ar[r]&\Hom_{\D}(W,Z')\ar[r]&\Hom_{\D}(W,X'[1])\ar[r]&\Hom_{\D}(W,Y'[1]).
\end{tikzcd}\]	
5項補題により,$\Hom_{\D}(W,\gamma)$が同型である.$W\in\D$は任意なので,米田の補題より$\gamma$は同型である.
\end{proof}

\begin{cor}\cite[p.246]{KS06}
三角圏 $\D$ において,射 $f \colon X \to Y$ に対して,
2つの完全三角形
\[
X \xrightarrow{f} Y \xrightarrow{g} Z \xrightarrow{h} X[1], \quad
X \xrightarrow{f} Y \xrightarrow{g'} Z' \xrightarrow{h'} X[1].
\]
が存在するとする.このとき,対象 $Z, Z'$ は同型であり,さらに完全三角形の間の同型射
\[
\begin{tikzcd}
X \arrow{r}{f} \arrow[equal]{d} & Y \arrow{r}{g} \arrow[equal]{d} & Z \arrow{r}{h} \arrow[dashed]{d}{\varphi} & X[1] \arrow[equal]{d} \\
X \arrow{r}{f} & Y \arrow{r}{g'} & Z' \arrow{r}{h'} & X[1].
\end{tikzcd}
\]
が存在する.このような$Z$を同型類から$1$つ選び,$f$の写像錘(mapping cone)と呼び,記号$\Cone(f)$と記す.
\end{cor}

\begin{cor}
三角圏 $\D$ における完全三角形
\[
X \xrightarrow{f} Y \xrightarrow{g} Z \xrightarrow{h} X[1].
\]
に対して,以下が成り立つ:

\begin{enumerate}
  \item $g$ が全射であることと$h = 0$であることは同値である.
  \item $g$ が単射ああることと$f = 0$であることは同値である.
  \item $h = 0$ ならば,$Y \cong X \oplus Z$ が成り立ち,この三角は
  \[
  X \xrightarrow{\iota} X \oplus Z \xrightarrow{\pi} Z \xrightarrow{0} X[1].
  \]
  に同型である(ただし $\iota$ は自然な包含、$\pi$ は射影).
\end{enumerate}
\end{cor}
\begin{proof}
	命題\ref{gf=0}より,$h\circ g = 0$であり,$g$が全射なら$h=0$.逆に,任意の$W\in\D$に対して,
\[
	\Hom_{\D}(X[1],W)\xrightarrow{\Hom_{\D}(h,W)}\Hom_{\D}(Z,W)\xrightarrow{\Hom_{\D}(g,W)}\Hom_{\D}(Y,W).
\]	
が完全列なので,$h=0$なら,$\Hom_{\D}(g,W)$は単射である.したがって$g$は全射である.2は双対的に示される.\\
3について,$h=0$とすると,
\begin{gather*}
	0\rightarrow\Hom_{\D}(Z,W)\rightarrow\Hom_{\D}(Y,W)\rightarrow\Hom_{\D}(X,W)\rightarrow 0,\\
	0\rightarrow\Hom_{\D}(W,X)\rightarrow\Hom_{\D}(W,Y)\rightarrow\Hom_{\D}(W,Z)\rightarrow 0.
\end{gather*}
が短完全列になることより,短完全列
\[
0\rightarrow X \xrightarrow{f} Y \xrightarrow{g} Z\rightarrow 0 .
\]
が分裂して$Y\simeq X\oplus Z$がわかる.
\end{proof}

\begin{prop}\cite[p.247]{KS06}
三角圏 $\D$ において,次の2つの完全三角形
\[
X \rightarrow Y \rightarrow Z \rightarrow X[1], \quad
X' \rightarrow Y' \rightarrow Z' \rightarrow X'[1].
\]
が与えられているとする.

また,射$g\colon Y\to Y'$ が与えられており,さらに
\[
\Hom_{\D}(X, Z') = \Hom_{\D}(X, Z'[-1]) = 0.
\]
が成り立つとする.

このとき,ある一意な射$f\colon X\to X'\quad h\colon Z\to Z'$が存在して,以下の図式が可換になる:
\[
\begin{tikzcd}
	X \arrow{r} \arrow[dotted]{d}{\exists !f} & Y \arrow{r} \arrow{d}{ g} & Z \arrow{r} \arrow[dotted]{d}{\exists !h} & X[1] \arrow[dotted]{d}{f[1]} \\
X' \arrow{r} & Y' \arrow{r} & Z' \arrow{r} & X'[1].
\end{tikzcd}
\]
すなわち,$(f,g,h)$ は完全三角形の間の射をなす.
\end{prop}
\begin{proof}
	コホモロジー的函手$\Hom_{\D}(X,-)$を2行目に作用させると,
	\[0=\Hom_{\D}(X,Z'[-1])\rightarrow \Hom_{\D}(X,X') \rightarrow \Hom_{\D}(X,Y') \rightarrow \Hom_{\D}(X,Z')=0.\]
	という完全列が得られ,可換にする一意的な射$f$の存在がわかる.同様に,$\Hom_{\D}(-,Z')$を1行目に作用させて,$\Hom_{\D}(X[1],Z')\simeq \Hom_{\D}(X,Z'[-1])$を用いれば,一意的な射$h$の存在もわかる.

\end{proof}
