\section{Bridgeland安定性条件}
\begin{defn}\cite{Bri07}
三角圏 $\D$ において,充満部分加法圏の族 $\{ \P(\phi) \}_{\phi \in \RR} \subset \D$が,次の条件を満たすとき,スライスという:
\begin{itemize}
  \item 任意の $\phi \in \RR$ に対して,$\P(\phi + 1) = \P(\phi)[1]$ が成り立つ.
  \item 任意の $\phi_1 > \phi_2$ に対して,$X_i \in \P(\phi_i)$ であるならば,$\Hom_{\D}(X_1, X_2) = 0$ である.
  \item 任意の対象 $X \in \D$ に対して,実数列 $\phi_1 > \phi_2 > \cdots > \phi_n$ と $Y_i \in \P(\phi_i)$ が存在し,次のような完全三角形の列による分解が存在する:
  \[
    \begin{tikzcd}[column sep=1.5em]
      0 \ar[r, equal] & X_0 \ar[rr] && X_1 \ar[r] \ar[ld] & \cdots \ar[r] & X_{n-2} \ar[rr] && X_{n-1} \ar[ld] \ar[rr] && X_n \ar[ld] \ar[r, equal] & X \\
      && Y_1 \ar[lu, dotted,swap, "{[1]}"'] &&&& Y_{n-1} \ar[lu, dotted,swap, "{[1]}"'] && Y_n \ar[lu, dotted,swap, "{[1]}"'].
    \end{tikzcd}
  \]
\end{itemize}
スライス$\P$と群準同型$Z\colon K(\D)\to \CC$の組$(\P,Z)$をBridgeland安定性条件という.ただし,$Z$は以下の条件をみたすものとする.\\
任意の$0\neq E\in\P(\phi)$に対して,ある$m(E)\in\RR_{>0}$存在して,
\[Z(E) = m(E)e^{i\pi\phi}.\]
と表すことができる.
\end{defn}


\begin{defn}\cite{Bri07}
$\A$をアーベル圏とする.群準同型:
\[
  Z \colon K(\A) \longrightarrow \CC.
\]
が次の条件を満たすとき,アーベル圏 $\A$ 上の安定性条件という:
\begin{itemize}
  \item[(i)] 任意の $X \in \A$,$X \neq 0$ に対して,$Z(X) \in \HH$ を満たす.ただし,
  \[
    \HH \coloneqq \{ re^{i\pi\phi} \in \CC \mid r > 0,\ 0 < \phi \le 1 \}.
  \]
  と定める.このとき,$Z(X) = |Z(X)|e^{i\pi\phi(X)}$ と書けるような $\phi(X) \in (0,1]$ を
  位相(phase)と呼ぶ.
  
  \item[(ii)] 任意の $X \in \A$ に対して,次のような完全列
  \[
    0 = X_0 \subset X_1 \subset \cdots \subset X_k = X.
  \]
  が存在して,各被対象 $Y_i \coloneqq X_i / X_{i-1}$ が $Z$ に関して$Z$-半安定であり,かつその位相が
  \[
    \phi(Y_1) > \phi(Y_2) > \cdots > \phi(Y_k).
  \]
  を満たす.このような列をHarder--Narasimhan分解(HN分解)と呼ぶ.
\end{itemize}

ここで,非零対象 $X \in \A$ が任意の非零部分対象 $0 \neq Y \subsetneq X$ に対して,
\[
   \phi(Y) \le  \phi(X).
\]
が成り立つとき,$X$は$Z$-半安定であるという.
\end{defn}

\begin{lemm}\cite{Bri07}\label{lemm:Hom is 0}
	$\D$をt-構造が与えられた三角圏,その核に安定性条件が与えられているとする.$X,Y\in\D^{\le 0}\cap\D^{\ge 0}$が半安定で$\phi(X)>\phi(Y)$であるとき,
	\[\Hom_{\D}(X,Y)=0.\]
	である.
\end{lemm}
\begin{proof}
	$f\in\Hom_{\D}(X,Y)$が$0$でないと仮定する.
	\[0\rightarrow\Ker f\rightarrow X \rightarrow \Im f \rightarrow 0.\]
	の完全列が存在して$X$が半安定対象であることから$\phi(\Im f)\ge\phi(X)$.$\Im f\neq 0$が$Y$の部分対象で$Y$も半安定であることから$\phi(Y)\ge\phi(\Im f)$.これらを合わせると$\phi(Y)\ge\phi(X)$.仮定に矛盾.
\end{proof}

\begin{prop}\cite{Bri07}
三角圏 $\D$ に対して,$\D$ 上の有界な $t$-構造と,その核のアーベル圏上の stability函数 $Z$ を与えることと,$\D$ 上のBridgeland安定性条件 $(Z, \P)$で,任意の $\phi \in \RR$ と $0 \neq X \in \P(\phi)$ に対して $Z(X) \in \RR_{>0} e^{i\pi\phi}$ を満たすものを与えることは同値である.
\end{prop}

\begin{proof}
		アーベル圏上の安定性条件$(Z,\A)\rightarrow$三角圏上のBridgeland安定性条件$\P$の構成\\
		各$0<\phi\le 1$に対して,
		\[\P(\phi)\coloneq \{X\in\A\mid X\colon Z\text{-半安定 }Z(X)\in\RR_{>0}e^{i\pi\phi}\}\cup \{0\}.\]
		$\phi\in\RR,\phi\in(k,k+1]$となる整数$k$をとって
		\[\P(\phi)\coloneq \P(\phi - k)[k].\]
		こう定めたとき,Bridgeland安定性条件を定めていることを確かめる.\\
		\bullet \ 任意の$\phi\in\RR$ に対して,$ \P(\phi + 1) = \P(\phi)[1]$は定義から従う.\\
		\bullet \ $\phi_1 > \phi_2$と$X_i\in\P(\phi_i)$対して,$\Hom_{\D}(X_1,X_2)=0$\\
		このとき$Y_1,Y_2\in\P((0,1])$を用いて,$X_1=Y_1[m],\ X_2=Y_2[n], (m>n\ \text{または}\ m=n,\ \phi(Y_1) > \phi(Y_2))$とかける.$m=n$のときは補題(\ref{lemm:Hom is 0})より$0$であることがわかる.$m>n$のとき
		$\Hom_{\D}(Y_1[m],Y_2[n])\simeq \Hom_{\D}(Y_1,Y_2[-(m-n)])$なので,$\Hom_{\D}(Y_1,Y_2[-k])=0,\ (k>0)$を示せばよいが,$Y_1\in\D^{\le 0},\ Y_2[-k]\in\D^{\ge k}$なのでt-構造の定義より$0$となることがわかる.\\
分解をもつことはt-構造の性質より,整数の列$k_1>k_2 \cdots > k_n$と分解
	\[
		\begin{tikzcd}[column sep=1.0em]
			0\ar[r,equal]&X_0\ar[rr]& & X_{1}\ar[r]\ar[ld] &\cdots\ar[r] &X_{n-2}\ar[rr]&&X_{n-1}\ar[ld]\ar[rr]&&X_n\ar[ld]\ar[r,equal]&X\\
									 &&A_{1}[k_1]\ar[lu,dotted,"{[1]}"] &&&&A_{n-1}[k_{n-1}]\ar[lu,dotted,"{[1]}"]&&A_n[k_n]\ar[lu,dotted,"{[1]}"].
		\end{tikzcd}
	\]
	$A_j\in \A$が存在する.各$A_j$には仮定よりHN分解が存在するので更に分解することで3つ目の条件が得られる.\\
	逆に$\P$が与えられたとき,$\F^{\perp}[1]=\D^{\ge 0}\coloneq\P((0,\infty)), \F=\D^{\le 0}\coloneq\P((-\infty,1])$がt-構造を定めることを確かめる.\\
\bullet\ $\D^{\le 0}\subset\D^{\le 1}\hspace{3mm}\D^{\ge 1}\subset\D^{\ge 0}\\ 
$\bullet\ $\D^{\ge 1}\subset (\D^{\le 0})^\perp$.つまり,任意の$X\in \D^{\le 0}, Y\in\D^{\ge 1}$に対して,$\Hom_{\D}(X,Y)=0$である\\
上記2つの条件については,スライスの定め方からしたがう.\\
\bullet\ 任意の$X\in\D$に対して
\[\tau_{\le 0}X\rightarrow X \rightarrow \tau_{\ge 1}X\rightarrow \tau_{\le 0}X[1].\]
	となるような$\tau_{\le 0}X\in\D^{\le 0},\ \tau_{\ge 1}X\in\D^{\ge 1}$が存在する.\\
 スライスの定義から,任意の$X\in \D$に対して,実数列$\phi_1 > \phi_2 > \cdots > \phi_n$ と $Y_i \in \P(\phi_i)$と以下の分解:
  \[
    \begin{tikzcd}[column sep=1.5em]
			0 \ar[r, equal] & X_0 \ar[rr] && X_1 \ar[rr]\ar[ld]&&X_2\ar[r]\ar[ld] & \cdots \ar[r] & X_{n-2} \ar[rr] && X_{n-1} \ar[ld] \ar[rr] && X_n \ar[ld] \ar[r, equal] & X \\
											&& Y_1 \ar[lu, dotted,swap, "{[1]}"'] &&Y_2 \ar[lu, dotted,swap, "{[1]}"']&&&& Y_{n-1} \ar[lu, dotted,swap, "{[1]}"'] && Y_n \ar[lu, dotted,swap, "{[1]}"'].
    \end{tikzcd}
  \]
	が存在する.八面体公理より,$Y_{1,2},Y_{n-1,n}$が存在して,
			\[
		\begin{tikzcd}
			X_0 \ar[r,]\ar[d,equal]& X_1\ar[r,]\ar[d,,swap]& Y_1\ar[d,,dotted] \ar[r]& X_0[1]\ar[d,equal]\\
			X_0 \ar[r,]\ar[d,,swap]& X_2\ar[r,]\ar[d,equal]& Y_{1,2}\ar[d,dotted] \ar[r]& X_0[1]\ar[d,]\\
			X_1 \ar[r,]\ar[d,,swap]& X_2\ar[r,]\ar[d,,swap]& Y_2 \ar[r]\ar[d,equal]& X_1[1]\ar[d,]\\
			Y_1 \ar[r,,dotted]& Y_{1,2}\ar[r,,dotted]& Y_2 \ar[r,,dotted]& Y_1[1]
		\end{tikzcd}\quad
\begin{tikzcd}
	X_{n-2} \ar[r,]\ar[d,equal]& X_{n-1}\ar[r,]\ar[d,,swap]& Y_{n-1}\ar[d,,dotted] \ar[r]& X_{n-2}[1]\ar[d,equal]\\
	X_{n-2}\ar[r,]\ar[d,,swap]& X_{n}\ar[r,]\ar[d,equal]& Y_{n-1,n}\ar[d,dotted] \ar[r]& X_{n-2}[1]\ar[d,]\\
	X_{n-1} \ar[r,]\ar[d,,swap]& X_n\ar[r,]\ar[d,,swap]& Y_n \ar[r]\ar[d,equal]& X_{n-1}[1]\ar[d,]\\
	Y_{n-1}\ar[r,,dotted]& Y_{n-1,n}\ar[r,,dotted]& Y_n \ar[r,,dotted]& Y_{n-1}[1].
		\end{tikzcd}
			\]
			それぞれ,二行目が完全三角形なので,
  \[
    \begin{tikzcd}[column sep=1.5em]
			0 \ar[r, equal] & X_0 \ar[rr] && X_2 \ar[rr]\ar[ld]&&X_3\ar[r]\ar[ld] & \cdots \ar[r] & X_{n-3} \ar[rr] && X_{n-2} \ar[ld] \ar[rr] && X_n \ar[ld] \ar[r, equal] & X \\
											&& Y_{1,2} \ar[lu, dotted,swap, "{[1]}"'] &&Y_3 \ar[lu, dotted,swap, "{[1]}"']&&&& Y_{n-2} \ar[lu, dotted,swap, "{[1]}"'] && Y_{n-1,n} \ar[lu, dotted,swap, "{[1]}"'].
    \end{tikzcd}
  \]
	とできる.また,四行目の完全三角形より$Y_1,Y_2\in\D^{\ge 1}$なら,$Y_{1,2}\in\D^{\ge 1}$であり,$Y_{n-1},Y_n\in\D^{\le 0}$なら,$Y_{n-1,n}\in\D^{\le 0}$がわかる.この操作を$\phi_k > 1$と$\phi_k\le 1$まで左右とも続ければ,
\[\tau_{\le 0}X\rightarrow X \rightarrow \tau_{\ge 1}X\rightarrow \tau_{\le 0}X[1].\]
	が完全三角形となるような$\tau_{\le 0}X\in\D^{\le 0},\ \tau_{\ge 1}X\in\D^{\ge 1}$の存在がわかる.
\end{proof}


\begin{prop}\cite{Bri07}
		$\A\colon$アーベル圏,群準同型$Z\colon K(\A)\to \CC$が次の条件を満たすとき,HN分解が存在する.つまり$Z$が安定性条件を定める.
		\begin{itemize}
			\item[(i)]
			全射の無限列
			\[X_1\twoheadrightarrow X_2\twoheadrightarrow X_3\twoheadrightarrow\cdots .  \]
		で$\phi (X_i)> \phi (X_{i+1})$となるものは存在しない.
	\item[(ii)]
			無限降下列
		\[X_1\supset X_2\supset X_3\supset \cdots  .\]
		で$\phi (X_{i+1}) > \phi(X_i)$となるものは存在しない.
	\end{itemize}
\end{prop}

\begin{proof}\hfill\\
	Step 1\hspace{5mm}\\
	\bullet 任意の対象$X\in\A$には$\phi(A) \ge \phi(X)$を満たす半安定な部分対象$A$が存在する.\\
	\bullet 任意の対象$X\in\A$には$\phi(X) \ge \phi(B)$を満たす半安定な対象$B$と全射$X\twoheadrightarrow B$が存在する.
	$X$が半安定ならOK.そうでないなら$\phi(X')>\phi(X)$となる部分対象$X'(\subsetneq X)$がとれる.\\
	これが無限回繰り返されると条件の(ii)に矛盾するので有限回でとまる.この極小となる$A$をとれば半安定である.\\
	同様に$X$が半安定なら全射$X\xrightarrow{\id}X$がとれてOK.そうでないとき,$X'(\subsetneq X)$で,$\phi(X')>\phi(X)$が存在して,$B'\coloneq X/X'$と定めると$\phi(X)\ge\phi(B')$であり,$X\twoheadrightarrow X/X'$がとれて,有限回でこの操作はとまるのでいずれ半安定になる.
\begin{center}
	\usetikzlibrary{arrows.meta}

\begin{tikzpicture}[scale=2, >=Stealth]

  % 原点を中心に設定
  \coordinate (O) at (0,0);

  % 各ベクトルの終点
  \coordinate (X') at (-0.5,0.7);     % Z(A)
  \coordinate (B') at (1.2,0.6);    % Z(X/A)
  \coordinate (X) at ($(X')+(B')$); % Z(X) = Z(A) + Z(X/A)

  % 軸
  \draw[->] (-1,0) -- (1.2,0) node[right] {\(\Re\)};
  \draw[->] (0,-0.2) -- (0,1.6) node[above] {\(\Im\)};

  % Z(A) ベクトル(青)
  \draw[->, thick,blue] (O) -- (X') node[midway, left] {\(Z(X')\)};

  % Z(X/A) ベクトル(緑):Aの先から
  \draw[->, thick] (X') -- (X) node[midway,above] {\(Z(B')\)};

  % Z(X) ベクトル(赤):原点から
  \draw[->, thick,red] (O) -- (X) node[midway, right] {\(Z(X)\)};

  % 原点
  \fill (O) circle (0.03);
  \node at (-0.15, -0.1) {\footnotesize 0};

\end{tikzpicture}
\end{center}

		Step 2\\
			$X\twoheadrightarrow B$が次の条件をみたすとき極大不安定商(mdq)と呼ぶ.\\
			\bullet $\phi(X)\ge\phi(B)$\\
			\bullet 任意の$X\twoheadrightarrow B'$に対して,$\phi(B')\ge \phi(B)$であり,$\phi(B')=\phi(B)$なら
			\[X\twoheadrightarrow B\twoheadrightarrow B'.\]
			と分解.\\
			このとき,任意の対象$X$はmdqを持つことがわかる.
	$X\twoheadrightarrow B'$において$B'$が半安定でないならStep 1から半安定な対象$B''$で$\phi(B')>\phi(B'')$と$B'\twoheadrightarrow B''$がとれるので,$B'$が半安定なときについて示せばよい.同様にmdqの$B$は半安定でなければならない.\\
$X$が半安定対象のとき,任意の全射$X\twoheadrightarrow B'$に対して,短完全列
\[0\rightarrow \Ker f\rightarrow X\xrightarrow{f} B'\rightarrow 0.\]
が存在するので$Z(B') = Z(X) - Z(\Ker f)$.安定対象なので,$\phi(\Ker f)\le \phi(X)$
\begin{center}
\begin{tikzpicture}[scale=2, >=Stealth]

  % 原点を中心に設定
  \coordinate (O) at (0,0);

  % 各ベクトルの終点
  \coordinate (Ker) at (0.7,0.9);     % Z(A)
  \coordinate (B') at (-1.2,0.4);    % Z(X/A)
  \coordinate (X) at ($(Ker)+(B')$); % Z(X) = Z(A) + Z(X/A)

  % 軸
  \draw[->] (-1,0) -- (1.2,0) node[right] {\(\Re\)};
  \draw[->] (0,-0.2) -- (0,1.6) node[above] {\(\Im\)};

  % Z(A) ベクトル(青)
  \draw[->, thick,blue] (O) -- (Ker) node[midway, right] {\(Z(\Ker f)\)};

  % Z(X/A) ベクトル(緑):Aの先から
  \draw[->, thick] (Ker) -- (X) node[midway,above] {\(Z(B')\)};

  % Z(X) ベクトル(赤):原点から
  \draw[->, thick,red] (O) -- (X) node[midway, left] {\(Z(X)\)};

  % 原点
  \fill (O) circle (0.03);
  \node at (-0.15, -0.1) {\footnotesize 0};

\end{tikzpicture}
\end{center}
したがって,$\phi(B')\ge\phi(X)$となり$X$が半安定対象のとき,$X\xrightarrow{\id}X$はmdqとなる.\\
そうでないとき,step1より$\phi(A)>\phi(X)$なる半安定対象$A\subsetneq X$と
\[0\rightarrow A\rightarrow X\rightarrow X'\rightarrow 0.\]
という完全列が存在する.$\phi(A)>\phi(X)>\phi(X')$となっている.$X'\twoheadrightarrow B$が$X'$のmdqとなっているとき,合成$X\twoheadrightarrow B$は$X$のmdqであることを示す.\\
\because $X\twoheadrightarrow B'$を半安定で$\phi(B')\le \phi(B)$となっているとすると
\[\phi(A)>\phi(X)>\phi(X')\ge\phi(B)\ge\phi(B').\]
$A,B'$は半安定対象であり,前の補題より$\Hom_{\D}(A,B')=0$
\[\begin{tikzcd}
	A \ar[r,hookrightarrow]\ar[rd,"0",swap,]& X\ar[r,twoheadrightarrow]\ar[d,twoheadrightarrow]& X/A = X'\ar[ld,dotted,two heads].\\
								 &B'
\end{tikzcd}\]
図式のように可換にする全射が普遍性から存在する.$X'\twoheadrightarrow B$がmdqなので$\mu(B')=\mu(B)$となり,mdqの条件より$X'\twoheadrightarrow B\twoheadrightarrow B'$と経由する.したがって$X\twoheadrightarrow B\twoheadrightarrow B'$が存在し,$X\twoheadrightarrow B$がmdqであることがわかる.\\
$X'$がmdqでない場合は$X$を$X'$に取り替えて議論を繰り返すことと条件(ii)から有限回でとまるのでmdqの存在がわかる.

		Step 3\\
		任意の$X\in\A$はHN分解をもつ.
	$0$でない$X\in\A$を任意にとる.$X$が半安定なら$0\subset X$がHN分解を与えている.そうでないとき,$X\twoheadrightarrow B^1$をmdqとして
	\[0\rightarrow X^1 \rightarrow X\rightarrow B^1\rightarrow 0.\]
	をとる.$X^1$が半安定であるなら$0\subsetneq X^1\subsetneq X$がHN分解になっている.($X/X^1\simeq B$でmdqの$B$が半安定であるため).$X^1$がmdqでないとき,$X^1\twoheadrightarrow B^2$をmdqとして
	\[0\rightarrow X^2\rightarrow X^1\rightarrow B^2\rightarrow 0.\]
	$Q=X/X^2$とすると,$X\twoheadrightarrow B^1$がmdqであることから$\phi(Q)\ge\phi(B^1)$であり,次の短完全列
	\[0\rightarrow B^2\rightarrow Q\rightarrow B^1\rightarrow 0.\]
	\begin{center}
	\begin{tikzpicture}[scale=2, >=Stealth]

		% 原点を中心に設定
		\coordinate (O) at (0,0);

		% 各ベクトルの終点
		\coordinate (B1) at (0.7,0.9);     % Z(A)
		\coordinate (B2) at (-1.2,0.4);    % Z(X/A)
		\coordinate (Q) at ($(B1)+(B2)$); % Z(X) = Z(A) + Z(X/A)

		% 軸
		\draw[->] (-1,0) -- (1.2,0) node[right] {\(\Re\)};
		\draw[->] (0,-0.2) -- (0,1.6) node[above] {\(\Im\)};

		\draw[->, thick,blue] (O) -- (B1) node[midway, right] {\(Z(B^1)\)};

		\draw[->, thick] (B1) -- (Q) node[midway,above] {\(Z(B^2)\)};

		\draw[->, thick,red] (O) -- (Q) node[midway, left] {\(Z(Q)\)};

		% 原点
		\fill (O) circle (0.03);
		\node at (-0.15, -0.1) {\footnotesize 0};

	\end{tikzpicture}
\end{center}
から$\phi(B^2)\ge\phi(Q)$が得られる.$\phi(B^2)=\phi(Q)=\phi(B^1)$と仮定すると,$X\twoheadrightarrow B^1$がmdqであることから$X\twoheadrightarrow B^1\twoheadrightarrow Q$となって,$B^1$と$Q$の双方向に全射があることから$B^1\simeq Q$となり,$B^2=0$.これは矛盾.したがって,$\phi(B^2)>\phi(B^1)$が成り立つ.この操作は条件(ii)より有限回でとまり,その商は半安定なのでHN分解を得る.
\end{proof}


%\begin{defn}\cite{Bri07}
%	$\D\colon$三角圏.$\D$上の安定性条件とは,$\D$の有界な$t$-構造のHeart $\A\subset\D$と$\A$と$\A$上の安定性条件
%	\[Z\colon K(\D)=K(\A)\to\CC\]
%	$(Z,\A)$のことである.
%\end{defn}
