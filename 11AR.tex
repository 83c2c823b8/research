\section{Auslander-Reiten理論}

\begin{defn}
	加法圏$\C$が次の条件を満たすとき,冪等完備という.\\
	任意の$X\in\C$と冪等射
	\[e\colon X\to X\quad (e^2 = e),\]
	に対して,
	\[p\circ i = \id_Y,\quad i\circ p = e.\]
	を満たす$Y\in\C$,$i\colon Y\to X,\ p\colon X\to Y$が存在する.
\end{defn}

\begin{prop}
$k$を体とする.$\T$を$k$-線型,$\Hom$-finite,冪等完備な三角圏とする.このとき,$\T$はKrull-Schmidt圏である.
\end{prop}
以下,$\T$を$k$-線型,$\Hom$-finite,冪等完備な三角圏,$\Serre$をそのSerre函手とする.


\begin{defn}
	$\C$をKrull-Schmidt圏,$f\in\Hom_{\C}(X,Y)$とする.このとき,\vspace{-3mm}
	\begin{itemize}
		\item[(1)]
			$f\circ g = \id_Y$となるような$g\colon Y\to X$が存在するとき,$f$はsplit epimorphismであるという.
		\item[(2)]
			$f$がsplit epimorphismでなく,任意のsplit epimorphismでない$g\colon Z\to Y$に対して,$g = f\circ h$となるような$h\colon Z\to X$が存在するとき,$f$をright almost splitという.
		\item[(3)]
			任意の$g\in\End(X)$に対して,$f=g\circ f$ならば$g$が同型射がなりたつならば,$f$をright minimalという.
		\item[(4)]
			$f$がright almost splitかつright minimalのとき,$f$をsink morphismという.
	\end{itemize}
\end{defn}

\begin{thm}
	任意の$X\in\ind\T$に対して,以下の完全三角形が存在する:\vspace{-3mm}
	\begin{itemize}
		\item 
		\[
		\begin{tikzcd}
			\Serre(X[-1])\ar[r,"g"]& Y\ar[r,"f"]& X\ar[r] & \Serre(X)
		\end{tikzcd}.
	\]
ただし,$f$はsink morphismで,$g$はsource morphismである.
\item
		\[
		\begin{tikzcd}
			X\ar[r,"f'"]& Y'\ar[r,"g'"]& \Serre^{-1}(X[1])\ar[r] & X[1]
		\end{tikzcd}.
	\]
ただし,$f$はsource morphismで,$g$はsink morphismである.
	\end{itemize}
\end{thm}

\begin{lemm}\label{lemm:free-C-action}
	安定性条件$\sigma = (Z,\P)$への複素数の作用は自由である.すなわち以下が成り立つ:\\
	$\xi,\eta\in\CC$としたとき,
	\[\sigma\cdot \xi = \sigma\cdot \eta\Longrightarrow  \xi= \eta.\]
\end{lemm}
\begin{proof}
	$\sigma = (Z,\P)$に対して,$\sigma\cdot (x+\sqrt{-1}y)= \sigma$のとき,$x+\sqrt{-1}y = 0 \quad (x,y\in\RR)$を示せばよい.$ (e^{-\sqrt{-1}\pi(x+\sqrt{-1}y)}Z,\P(\bullet + x))$であるので,任意の$\phi\in\RR$に対して,
	\[\P(\phi + x) = \P(\phi).\]
	が成り立つ,したがって,$x=0$である.また,$e^{\pi y}Z = Z$なので$e^{\pi y} =1$.したがって,$y =0$
\end{proof}

\begin{thm}
	$\overrightarrow{\Delta}$をADE型箙,$\D = D^b(\mod k\overrightarrow{\Delta})$とし,そのSerre函手を$\Serre$
	とする.このとき,$\Serre$に関してGepner型である安定性条件が存在し,$\CC$作用を除いて一意である.
\end{thm}
\begin{proof}
	$\sigma = (Z,\P)$を$\Serre$に関してGepner型であるとすると,$\xi\in\CC$を用いて
\[\Serre\cdot \sigma = \sigma\cdot\xi .\]
したがって,$\Serre^h\cdot\sigma = \sigma\cdot h\xi$となる.また,AR理論より,
	\[\Serre^h \simeq (\tau[1])^h \simeq \tau^h[h] \simeq [h-2].\]
したがって,
	\[\sigma \cdot (h-2)=[h-2]\cdot \sigma = \sigma\cdot h\xi.\]
	(補題\ref{lemm:free-C-action})より,$h-2 = h\xi$,すなわち,$\xi = 1 -\frac{2}{h}$がしたがう.
\end{proof}

