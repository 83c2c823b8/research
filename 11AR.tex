\section{Auslander-Reiten理論}

\begin{defn}
	加法圏$\C$が次の条件を満たすとき,冪等完備という.\\
	任意の$X\in\C$と冪等射
	\[e\colon X\to X\quad (e^2 = e),\]
	に対して,
	\[p\circ i = \id_Y,\quad i\circ p = e.\]
	を満たす$Y\in\C$,$i\colon Y\to X,\ p\colon X\to Y$が存在する.
\end{defn}

\begin{prop}
$k$を体とする.$\T$を$k$-線型,$\Hom$-finite,冪等完備な三角圏とする.このとき,$\T$はKrull-Schmidt圏である.
\end{prop}
以下,$\T$を$k$-線型,$\Hom$-finite,冪等完備な三角圏,$\Serre$をそのSerre函手とする.


\begin{defn}
	$\C$をKrull-Schmidt圏,$f\in\Hom_{\C}(X,Y)$とする.このとき,\vspace{-3mm}
	\begin{itemize}
		\item[(1)]
			$f\circ g = \id_Y$となるような$g\colon Y\to X$が存在するとき,$f$はsplit epimorphismであるという.
		\item[(2)]
			$f$がsplit epimorphismでなく,任意のsplit epimorphismでない$g\colon Z\to Y$に対して,$g = f\circ h$となるような$h\colon Z\to X$が存在するとき,$f$をright almost splitという.
		\item[(3)]
			任意の$g\in\End(X)$に対して,$f=g\circ f$ならば$g$が同型射がなりたつならば,$f$をright minimalという.
		\item[(4)]
			$f$がright almost splitかつright minimalのとき,$f$をsink morphismという.
	\end{itemize}
\end{defn}

\begin{thm}
	任意の$X\in\ind\T$に対して,以下の完全三角形が存在する:\vspace{-3mm}
	\begin{itemize}
		\item 
		\[
		\begin{tikzcd}
			\Serre(X[-1])\ar[r,"g"]& Y\ar[r,"f"]& X\ar[r] & \Serre(X)
		\end{tikzcd}.
	\]
ただし,$f$はsink morphismで,$g$はsource morphismである.
\item
		\[
		\begin{tikzcd}
			X\ar[r,"f'"]& Y'\ar[r,"g'"]& \Serre^{-1}(X[1])\ar[r] & X[1]
		\end{tikzcd}.
	\]
ただし,$f$はsource morphismで,$g$はsink morphismである.
	\end{itemize}
\end{thm}
