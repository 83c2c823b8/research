\section{$\qgr R$の定義}

以下$R$を可換環とする.
\begin{defn}
$\A$ をアーベル圏,$\B \subset \A$ を Serre 部分圏とする.すなわち,$\B$ は次の条件を満たす:
\begin{itemize}
  \item 任意の短完全列
  \[
  0 \to X \to Y \to Z \to 0.
  \]
  が $\A$ にあり,$Y$ が $\B$ に属するとき,$X$ および $Z$ も $\B$ に属する(逆も同様).
\end{itemize}
このとき,$\B$ による $\A$ のSerre 商圏(Serre quotient category)$\A/\B$ は次のように定義される:
\begin{itemize}
  \item 対象は $\A$ の対象と同じである.:
		\[\Ob(\A/\B)\coloneq \Ob(\A).\]
  \item 射は以下のように定義される:
  \[
  \Hom_{\A/\B}(X, Y) := \varinjlim_{\substack{X' \subset X\\ X/X' \in \B}} \varinjlim_{\substack{Y' \subset Y\\ Y' \in \B}} \Hom_{\A}(X', Y/Y').
  \]
\end{itemize}
\end{defn}
	
\begin{defn}
次数付き環 \( R = \bigoplus_{n \ge 0} R_n \) 上の次数付き加群 \( M \in \gr R \) に対して,元 \( x \in M \) が 捻れ元(torsion element) であるとは,ある整数 \( n \gg 0 \) が存在して,\( R_{\ge n} \cdot x = 0 \) となることをいう.ここで \( R_{\ge n} = \bigoplus_{m \ge n} R_m \) とする.

加群 \( M \) が 捻れ加群(torsion module) であるとは,すべての元 \( x \in M \) が捻れ元であることである.このような加群全体のなす部分圏を \(\tors R\) で表す.
\end{defn}

\begin{prop}
次数付き環 \( R = \bigoplus_{n \ge 0} R_n \) に対し,その上の次数付き右$R$-加群全体のアーベル圏 \(\gr R\) において,torsion 加群全体からなる部分圏 \(\tors R\) は Serre部分圏である.
\end{prop}

\begin{defn}\cite{AZ94}
\(
R = \bigoplus_{n \ge 0} R_n
\) を次数付きネーター$\CC$-代数とする.\vspace{-3mm}
\begin{itemize}
  \item $\gr R$ は $\mathbb{Z}$-次数付き有限生成$R$-加群のアーベル圏とする.
  \item $\tors R$ は $\gr R$ のうち,ねじれ加群全体とする.
  \item $\qgr R$ は $\gr R$ の Serre 部分圏 $\tors R$ による商圏:$\qgr R := \gr R / \tors R$ と定義する.
\end{itemize}
\end{defn}

\begin{thm}[Serreの定理 {\cite{Serre55}}]
\( A = \bigoplus_{n \ge 0} A_n \) を次数1の元で生成される \(\ZZ\)次数付き ネーター環とし,\( A_0 \) もネーター環とする.このとき,スキーム \( X = \operatorname{Proj} A \) に対して,以下のアーベル圏の圏同値が存在する:
\[
\Coh X \simeq \qgr A.
\]
この圏同値は,\( \F \in \Coh X \) に対して次のように定まる加群
\[
\Gamma_\star(\F) = \bigoplus_{d \in \ZZ} H^0(X, \F \otimes \mathcal{O}_X(d))
\]
をとして対応させることにより与えられる.
\end{thm}

\begin{prop}\cite{GL87}
$\CC$ を基礎体とし,$a_0, \dots, a_n \in \mathbb{Z}_{>0}$ とする.重み付き多項式環 $R = \CC[x_0, \dots, x_n]$ に対し,対応する Deligne–Mumford スタックを
\[
\mathcal{X} := [(\Spec R \setminus \{0\}) / \CC^*].
\]
と定める.このとき,スタック $\mathcal{X}$ 上の連接層のアーベル圏は次と同値である:
\[
\Coh(\mathcal{X}) \simeq \qgr R := \gr R / \tors R.
\]
\end{prop}



\begin{prop}[\cite{AZ94}]
$R = \CC[x_0, \dots, x_n]$ を重み付き多項式環とする.このとき,$\qgr R$ における射影対象は,次のような有限直和
\[
\bigoplus_{i=1}^r \mathcal{O}(n_i).
\]
で表される.
ここで,$\mathcal{O}(n) := \pi(R(n))$ は $R(n)$ の $\qgr R$ における像である.
\end{prop}


\begin{exmp}
$R = \CC[x, y]$ を $\ZZ$次数付き多項式環とし,$\deg(x) = 1$,$\deg(y) = 2$ とする.このとき,$R$ は重み付き射影直線 $\mathbb{P}(1,2)$ に対応し,次のような圏同値が成り立つ:
\[
\qgr R := \gr R / \tors R \simeq \Coh(\mathbb{P}(1,2)).
\]
\[
R := \CC[x, y], \quad \deg(x) = 1,\quad \deg(y) = 2.
\]

各\(n \in \ZZ_{\ge 0}\) に対して次数成分 \(R_n\) は次のように与えられる:
\[
\begin{aligned}
R_0 &= \CC \\
R_1 &= \CC x \\
R_2 &= \CC x^2 \oplus \CC y \\
R_3 &= \CC x^3 \oplus \CC x y \\
R_4 &= \CC x^4 \oplus \CC x^2 y \oplus \CC y^2 \\
&\vdots 
\end{aligned}
\]

このとき,次の列
\[
(\mathcal{O},\ \mathcal{O}(1),\ \mathcal{O}(2)),
\]
は \(\D^b(\Coh(\mathbb{P}(1,2)))\) における強例外生成列(\ref{defn:strong exceptional collection})であり,その直和
\[
T := \mathcal{O} \oplus \mathcal{O}(1) \oplus \mathcal{O}(2),
\]
は傾斜対象(\ref{defn:tilting object})となる.したがって,導来函手
\[
\RHom(T, -) \colon \D^b(\Coh(\mathbb{P}(1,2))) \longrightarrow \D^b(\mod \End(T)).
\]
は三角同値を与える.
\end{exmp}

\begin{defn}[Koszul 環]
体 $\Bbbk$ 上の $\mathbb{Z}_{\geq 0}$-次数付き結合多元環 $R = \bigoplus_{i \geq 0} R_i$ がKoszul であるとは,単純加群 $\Bbbk = R / R_{>0}$ が射影分解
\[
\cdots \to P^n \to \cdots \to P^1 \to P^0 \to \Bbbk \to 0
\]
において,各 $P^n$ が次数付き自由加群であり,その生成元がすべて次数 $n$ にあるときに言う.すなわち,
\[
\operatorname{Ext}^n_R(\Bbbk, \Bbbk)_m = 
\begin{cases}
\Bbbk & \text{if } m = n, \\
0 & \text{if } m \ne n
\end{cases}
\quad \text{for all } n, m \in \mathbb{Z}.
\]
が成り立つとき,$R$ は Koszul 環である.
\end{defn}

\begin{defn}[Artin–Schelter regular 環]
$\Bbbk$ 上の $\mathbb{Z}_{\geq 0}$-次数付き結合多元環 $R = \bigoplus_{i \geq 0} R_i$ が Artin--Schelter regular(AS-regular)であるとは,次の三条件を満たすときに言う:
\begin{enumerate}
  \item $R$ は右および左ともに有限生成の $\Bbbk$-代数である.
  \item $R$ の大域次元 $\gldim(R) = d < \infty$.
  \item Gorenstein 条件:ある整数 $l$ が存在して
  \[
  \operatorname{Ext}^i_R(\Bbbk, R) = 
  \begin{cases}
  \Bbbk(l) & \text{if } i = d, \\
  0 & \text{if } i \ne d
  \end{cases}
  \quad \text{および} \quad
  \operatorname{Ext}^i_{R^\mathrm{op}}(\Bbbk, R) = 
  \begin{cases}
  \Bbbk(l) & \text{if } i = d, \\
  0 & \text{if } i \ne d
  \end{cases}
  \]
  が成り立つ(ここで $\Bbbk(l)$ は次数 $l$ だけシフトされた $\Bbbk$ である).
\end{enumerate}
\end{defn}

\begin{thm}
\label{thm:tilting_qgr}
次数付き環 \(R = \bigoplus_{n \ge 0} R_n\) が次の条件を満たすとする:

\begin{enumerate}
  \item \(R\) は \(\CC\)-線型で右ネーター.
  \item \(R_0 = \CC\).
  \item \(R\) は Koszul かつ AS-regular,大域次元を \(d\) とする.
\end{enumerate}

このとき,対象
\[
T := \bigoplus_{i=0}^{d-1} \pi(R(i)) \in \qgr R
\]
は \(\D^b(\qgr R)\) における傾斜対象(\ref{defn:tilting object})であり,導来函手
\[
\RHom_{\qgr R}(T, -) \colon \D^b(\qgr R) \longrightarrow \D^b(\mod \End_{\qgr R}(T))
\]
は三角同値を与える.
\end{thm}

\begin{prop}\cite{Ri89}
\label{prop:t_structure_from_tilting}
定理 \ref{thm:tilting_qgr} の状況において,
\[
\RHom_{\qgr R}(T, -) \colon \D^b(\qgr R) \xrightarrow{\sim} \D^b(\mod A), \quad A := \End_{\qgr R}(T).
\]
が三角同値であるとする.このとき,
\[
\A := \left\{ X \in \D^b(\qgr R) \mid \RHom_{\qgr R}(T, X) \in \mod A \subset \D^b(\mod A) \right\}.
\]
は \(\D^b(\qgr R)\) 上の \(t\)-構造の核となる.
\end{prop}

