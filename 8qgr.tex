\section{$\qgr R$の定義}

以下$R$を可換環とする.
\begin{defn}
$\A$ をアーベル圏,$\B \subset \A$ を Serre 部分圏とする.すなわち,$\B$ は次の条件を満たす:
\begin{itemize}
  \item 任意の短完全列
  \[
  0 \to X \to Y \to Z \to 0.
  \]
  が $\A$ にあり,$Y$ が $\B$ に属するとき,$X$ および $Z$ も $\B$ に属する(逆も同様).
\end{itemize}
このとき,$\B$ による $\A$ のSerre 商圏(Serre quotient category)$\A/\B$ は次のように定義される:
\begin{itemize}
  \item 対象は $\A$ の対象と同じである.:
		\[\Ob(\A/\B)\coloneq \Ob(\A).\]
  \item 射は以下のように定義される:
  \[
  \Hom_{\A/\B}(X, Y) := \varinjlim_{\substack{X' \subset X\\ X/X' \in \B}} \varinjlim_{\substack{Y' \subset Y\\ Y' \in \B}} \Hom_{\A}(X', Y/Y').
  \]
\end{itemize}
\end{defn}
	
\begin{defn}
	 \( R = \bigoplus_{n \ge 0} R_n \)を次数付き環,\( M \in \gr R \)を$R$上の次数付き加群とする. $M$の元\( x \in M \) に対して,ある整数 \( n \gg 0 \) が存在して,\( R_{\ge n} \cdot x = 0 \) となるとき, 捻れ元(torsion element) であるという.ここで \( R_{\ge n} = \bigoplus_{m \ge n} R_m \) とする.

すべての元 \( x \in M \) が捻れ元である加群 \( M \)のことを捻れ加群(torsion module) であるという.このような加群全体のなす充満部分圏を \(\tors R\) で表す.
\end{defn}

\begin{prop}
次数付き環 \( R = \bigoplus_{n \ge 0} R_n \) に対し,その上の次数付き右$R$-加群全体のアーベル圏 \(\gr R\) において,torsion 加群全体からなる部分圏 \(\tors R\) は Serre部分圏である.
\end{prop}

\begin{defn}\cite{AZ94}
\(
R = \bigoplus_{n \ge 0} R_n
\) を次数付きネーター$\CC$-代数とする.\vspace{-3mm}
\begin{itemize}
  \item $\gr R$ は $\mathbb{Z}$-次数付き有限生成$R$-加群のアーベル圏とする.
  \item $\tors R$ は $\gr R$ のうち,ねじれ加群全体とする.
  \item $\qgr R$ は $\gr R$ の Serre 部分圏 $\tors R$ による商圏:$\qgr R := \gr R / \tors R$ と定義する.
\end{itemize}
\end{defn}

\begin{defn}\cite{GL87}
重み付き射影直線(weighted projective line)とは,以下のデータによって定まる射影曲線である:

\begin{itemize}
  \item 正の整数からなる列 $A = (a_0, a_1, \dots, a_r)$,
  \item $\mathbb{P}^1(k)$ の互いに異なる点の列 $\Lambda = (\lambda_0, \lambda_1, \dots, \lambda_r)$,
\end{itemize}

ただし,通常 $\lambda_0 = \infty,\ \lambda_1 = 0,\ \lambda_2 = 1$ と正規化する.次に,アーベル群
\[
	L_{A} \coloneq \ZZ \vec{x_1}\oplus\cdots\oplus\ZZ \vec{x_r}\oplus \ZZ\vec{c}\Big{/}\langle\vec{c} -a_i\vec{x_i}\mid i=1,\ldots,r\rangle.
\]
を定義し,これを次数付き群とする.多項式環: 
\[
	S = k[X_0, X_1, \dots, X_n], \quad \deg X_i = \vec{x}_i. 
\]
において,以下の $L_A$-斉次イデアル:
\[
	I_{A,\Lambda} = \left( X_i^{a_i} - X_1^{a_1} + \lambda_i X_0^{a_0} \ \middle|\ i = 2, \dots, n \right).
\]
を用いて商環
\[
	R_{A,\Lambda} = k[X_0, X_1, \dots, X_n]\Big{/}I_{A,\Lambda} = k[x_0,x_1,\ldots, x_n],\quad \deg(x_i)=\vec{x_i}.
\]
を定義する.
\end{defn}

$L_A$は階数$1$のアーベル群であり,各次数$\vec{\ell}\in L_A$は一意的に,
\[\vec{\ell} = \sum_{i=0}^n \ell_i\vec{x_i} + \ell\vec{c}\quad (0\le \ell_i < p_i,\ \ \ell \in \ZZ).\]
とかける.
ここで,
\[\vec{\omega} \coloneq (n-1)\vec{c} - \sum_{i=0}^n\vec{x_i}.\]
と$\vec{\omega}$を定義し,双対化元(dualizing element)と呼ぶ.また,
\[R(A)= \bigoplus_{\ell=0}^\infty R_{\ell},\quad R_\ell = S_{\ell\vec{c}}.\]
を$R_{A,\Lambda}$の核(core)と呼ぶ.

\begin{defn}\cite{GL87}
	Serreの定理より,重み付き射影直線 $\mathbb{X}$に対応する $L_A$-次数付き環 $R_{A,\Lambda}$ に対して,次のように連接層の圏$\Coh(\mathbb{X})$ を定義する:

\[
\Coh(\mathbb{X}) := \qgr R_{A,\Lambda} = \gr R_{A,\Lambda}\Big{/}\tors R_{A,\Lambda}.
\]

ここで,
\begin{itemize}
	\item $\gr R_{A,\Lambda}$ は $L_A$-次数付き有限生成$R_{A,\Lambda}$-加群の圏,
	\item $\tors R_{A,\Lambda}$ は $\gr R_{A,\Lambda}$のねじれ加群全体の圏.
\end{itemize}
さらに,$\vec{x} \in L(\mathbf{p})$ に対して,ひねり(twist)とは,$S$-加群 $M$ に対して次のように定義される加群 $M(\vec{x})$ を与える操作である:
\[
M(\vec{x})_{\vec{y}} := M_{\vec{x} + \vec{y}} \quad (\vec{y} \in L(\mathbf{p})).
\]
$(M,\vec{x})\to M(\vec{x})$により,各圏に対して$L_A$作用が定まる.
\end{defn}

\begin{thm}[Krull--Schmidt 性]\cite{GL87}
重み付き射影直線 $\mathbb{X} $ に対して,連接層のアーベル圏
\[
\Coh(\mathbb{X}) = \qgr R_{A,\Lambda} = \gr R_{A,\Lambda}\Big{/}\tors R_{A,\Lambda}.
\]
は Krull--Schmidt 圏である.すなわち,任意の対象 $\mathcal{F} \in \Coh(\mathbb{X})$ は直既約対象の有限直和に分解でき,その分解は同型と順序を除いて一意である:
\[
\mathcal{F} \cong \bigoplus_{i=1}^n \mathcal{F}_i \quad (\text{各} \mathcal{F}_i \text{が既約対象}).
\]
\end{thm}

\begin{lemm}
	重み付き射影直線 $\mathbb{X}$において,各直線束$L$は$\Coh(\XX)$における例外対象である.
\end{lemm}
\begin{proof}
	$L = \mathcal{O}(\vec{x})$とかけるので,
	\[\Hom(\mathcal{O}(\vec{x}),\mathcal{O}(\vec{x}))=S_0=k.\]
	である.
\end{proof}

\begin{thm}\cite{GL87}
重み付き射影直線 $\mathbb{X}$ において,次の加群
\[
T := \bigoplus_{0 \le \vec{x} \le \vec{c}} \mathcal{O}_{\mathbb{X}}(\vec{x}).
\]
をとると,これは $\Coh(\mathbb{X})$ における傾対象である.
\end{thm}

\begin{thm}[Serre 双対性]\cite{GL87}
重み付き射影直線 $\mathbb{X}$ において,任意の連接層 $\mathcal{F}, \mathcal{G} \in \Coh(\mathbb{X})$ に対して,次の自然なベクトル空間の同型が成り立つ:
\[
\Ext^1(\mathcal{F}, \mathcal{G})^\vee \cong \Hom(\mathcal{G}, \mathcal{F}(\vec{\omega})).
\]
\[
\RHom(\F,\G)^\vee \simeq \RHom(\G,\F(\vec{\omega}))[1]?\]
ここで,
\begin{itemize}
	\item $(-)^\vee := \Hom_{\CC}(-, \CC)$ は $\CC$ 上の線型双対,
  \item $\vec{\omega} = (n - 1)\vec{c} - \sum_{i=0}^n \vec{x}_i$ は dualizing element(双対化元)である.
\end{itemize}
この同型は $\mathcal{F}, \mathcal{G}$ に関して函手的である.
\end{thm}

\begin{exmp}
$R = \CC[x, y]$ を $\ZZ$次数付き多項式環とし,$\deg(x) = 1$,$\deg(y) = 2$ とする.このとき,$R$ は重み付き射影直線 $\mathbb{P}(1,2)$ に対応し,次のような圏同値が成り立つ:
\[
\qgr R := \gr R / \tors R \simeq \Coh(\mathbb{P}(1,2)).
\]
\[
R := \CC[x, y], \quad \deg(x) = 1,\quad \deg(y) = 2.
\]

各\(n \in \ZZ_{\ge 0}\) に対して次数成分 \(R_n\) は次のように与えられる:
\[
\begin{aligned}
R_0 &= \CC \\
R_1 &= \CC x \\
R_2 &= \CC x^2 \oplus \CC y \\
R_3 &= \CC x^3 \oplus \CC x y \\
R_4 &= \CC x^4 \oplus \CC x^2 y \oplus \CC y^2 \\
&\vdots 
\end{aligned}
\]

このとき,次の列
\[
(\mathcal{O},\ \mathcal{O}(1),\ \mathcal{O}(2)),
\]
は \(\D^b(\Coh(\mathbb{P}(1,2)))\) における強例外生成列(\ref{defn:strong exceptional collection})であり,その直和
\[
T := \mathcal{O} \oplus \mathcal{O}(1) \oplus \mathcal{O}(2),
\]
は傾斜対象(\ref{defn:tilting object})となる.したがって,導来函手
\[
\RHom(T, -) \colon \D^b(\Coh(\mathbb{P}(1,2))) \longrightarrow \D^b(\mod \End(T)).
\]
は三角同値を与える.
\end{exmp}

