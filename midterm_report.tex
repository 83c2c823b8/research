\RequirePackage{luatex85}
\documentclass[leqno]{ltjsarticle}
\usepackage{luatexja-fontspec}
\usepackage[top=20truemm,bottom=25truemm,left=25truemm,right=25truemm]{geometry}
\usepackage{luatexja} 
\usepackage{multicol,amsmath,amssymb,mathtools,ascmac,amsthm,amscd,physics,comment,dcolumn,titlesec,mathrsfs,mystyle,tikz-cd}
\usetikzlibrary{arrows.meta}
\titleformat*{\section}{\Large\bfseries}
%\setlength{\parindent}{0pt}
\pagestyle{empty}
%\everymath{\displaystyle}
\begin{document}
\title{研 究 概 要 報 告 書\\
重み付き射影直線上の安定性条件の空間}
\date{}
\author{大阪大学大学院理学研究科数学専攻\\山本 雄大}
\maketitle

安定性の概念はベクトル束のslope安定性から始まり,モジュライ空間を構成するための条件として導入された.その後,Bridgeland により三角圏上の安定性条件が定式化され,その空間が複素多様体となることが示された.これにより安定性は個々の対象の性質ではなく,圏全体の構造を与える幾何的データとして理解されるようになった.重み付き射影直線に付随する導来圏では例外生成列の存在や傾理論により圏の構造が具体的に扱えること,さらに半直交分解が明示的に記述できる場合があることから,安定性条件を調べるための良い対象となる.しかし安定性条件の空間がどのような壁構造を持つかは十分に記述されていない.本研究では重み付き射影直線の導来圏において,核の取り方と壁の生成の関係を具体例を通して理解することを目的とする.\par
まず,重み付き射影直線に関する基礎事項を整理し,連接層の圏とその導来圏の構造を確認した.特に半直交分解を取ることで現れるADE 型箙の導来圏との関係を調べることにした.その上でADE 箙の導来圏における安定性条件の空間について既存結果を確認し,半直交分解に伴う安定性条件の制限と誘導の関係を調べている.さらに,部分圏で定義された安定性条件を貼り合わせることで全体の圏上の安定性条件がどのように構成されるかを検討している.現在は特にゲプナー型安定性条件の構成と,それが上記の貼り合わせとどのように関係するかを調べている段階にある.\par
安定性条件の定義を三角圏と t-structure の対応から整理し,安定性条件が核と中心電荷の組として構成されることを理解した.さらに半安定対象の位相と Harder–Narasimhan 分解の存在を確認し,安定性条件が対象の分解構造を与えるものであることを把握した.また安定性条件の空間に距離が入り,変形を考える対象になることを理解した.特に重み付き射影直線の導来圏における Serre 不変性および Gepner 型安定性条件の意味を整理し,その構成を確認した.さらに,ADE型箙の表現の圏の導来圏では,複素数による自然な作用を除けば,Gepner 型安定性条件が本質的に一意に定まることを確かめた.\par
一方で,三角圏の半直交分解が与えられたとき,全体の圏の安定性条件と各成分部分圏の安定性条件の間の一般的な対応はまだ理解できていない.特に,各部分圏上の安定性条件から全体の圏上の安定性条件がどのように構成されるかは明確ではない.現在の理解では,部分圏で定義された安定性条件を適切に整合させることで,全体の圏の安定性条件が張り合わせにより得られるのではないかと期待されるが,その具体的条件や必要十分条件は分かっていない.そのため,まず変異によって移り合う部分圏に対応する安定性条件を比較し,それらを同時に含むような安定性条件を構成することを考えている.\par
この問題が難しい理由は,安定性条件が局所的な圏構造からは決まらない点にある.半直交分解は射の消滅条件を与えるが,安定性条件は拡大構造を用いて定義されるため,各部分圏の核から全体の核を単純に再構成することが出来ない.特に半安定性は部分対象の位相の比較により決まるが,対象の分解は部分圏をまたいで現れるため,部分圏ごとの安定性条件を与えても全体での Harder–Narasimhan 分解が保たれるとは限らない.このため安定性条件は単純な貼り合わせとして構成できず,その整合条件を記述する必要がある.\par
その手始めとして,重み付き射影直線の導来圏と,その半直交分解の成分として現れる ADE 型箙の表現の圏の導来圏においてこの問題を具体的に調べる.特にミラー対称性から期待される構造を手がかりとして,全体の圏と部分圏の安定性条件の関係を同定することを目標とする.\par
安定性条件が構成されると,三角圏の対象が Harder–Narasimhan 分解を持ち,圏を分解可能な形で扱えるようになる.さらに安定性条件の空間は連続的に変形できる幾何的対象となり,壁越えに伴う組合せ的構造や自己同値群の作用を通して圏に付随する不変量を取り出すことが可能になる.
\end{document}
