\RequirePackage{luatex85}
\documentclass[leqno,11pt]{ltjsarticle}
\usepackage{luatexja-fontspec}
\usepackage[top=10truemm,bottom=15truemm,left=20truemm,right=20truemm]{geometry}
\usepackage{luatexja} 
\usepackage{multicol,amsmath,amssymb,mathtools,ascmac,amsthm,amscd,physics,comment,dcolumn,titlesec,mathrsfs,tikz-cd}
\usetikzlibrary{arrows.meta}
\titleformat*{\section}{\Large\bfseries}
%\setlength{\parindent}{0pt}
\pagestyle{empty}
%\everymath{\displaystyle}
\begin{document}
\title{研 究 概 要 報 告 書\\
重み付き射影直線上の安定性条件の空間}
\date{}
\author{大阪大学大学院理学研究科数学専攻\\山本 雄大}
\maketitle

代数幾何や表現論では,シフト,完全三角形と呼ばれるホモロジー代数を展開するのに必要な情報を持つ三角圏を用いて,様々な理論が統一的に記述される.三角圏は代数幾何や表現論において基本的な役割を果たすが,一般には対象の分解構造を直接扱うことが難しい.この困難を克服する枠組みとして導入されたのが Bridgeland の安定性条件である.安定性条件は三角圏に位相構造と分解理論を与え,任意の対象が Harder–Narasimhan 分解を持つようにする.その結果,三角圏の対象を「安定な構成要素」に分解し,段階的に理解することが可能になる.さらに安定性条件全体は複素多様体を成し,圏に付随する幾何的パラメータ空間として振る舞う.この空間の構造を理解することは,圏の内部構造や不変量の抽出に直結すると期待される.\\ \par

本研究で扱う軌道体射影直線(weighted projective line)は,射影直線に有限個の点で重みを付与した Deligne–Mumford 型のスタックである.これは,幾何学的にはいくつかの点が円錐の頂点のようにとがった形をしている空間と考えることができる.この対象は Geigle–Lenzing により導入され,その連接層の圏は有限次元代数の表現論と深く関係することが知られている.とくにその導来圏は例外対象列や半直交分解を通して ADE 型箙の導来圏と結びつく.とくに安定性条件の空間がどのような構造を持つかを理解することは,圏の自己同値や半直交分解との関係を明らかにする上で重要である.\par

安定性の概念はベクトル束のスロープ安定性から始まり,モジュライ空間を構成するための条件として導入された.その後,Bridgeland により三角圏上の安定性条件が定式化され,その空間が複素多様体となることが示された.これにより安定性は個々の対象の性質ではなく,圏全体の構造を与える幾何的データとして理解されるようになった.重み付き射影直線に付随する導来圏では例外生成列の存在や傾理論により圏の構造が具体的に扱えること,さらに半直交分解が明示的に記述できる場合があることから,安定性条件を調べるための良い対象となる.\\ \par

以上の背景を踏まえ,自身の作業として次の点を行った.まず,重み付き射影直線に関する基礎事項を整理し,連接層の圏とその導来圏の構造を確認した.特に半直交分解を取ることで現れるADE 型箙の導来圏との関係を調べることにした.その上でADE 箙の導来圏における安定性条件の空間について既存結果を確認し,半直交分解に伴う安定性条件の制限と誘導の関係を調べている.さらに,部分圏で定義された安定性条件を貼り合わせることで全体の圏上の安定性条件がどのように構成されるかを検討している.現在は特にゲプナー型安定性条件の構成と,それが上記の貼り合わせとどのように関係するかを調べている段階にある.\par

安定性条件の定義を三角圏と t-structure の対応から整理し,安定性条件が核と中心電荷の組として構成されることを理解した.さらに半安定対象の位相と Harder–Narasimhan 分解の存在を確認し,安定性条件が対象の分解構造を与えるものであることを把握した.また安定性条件の空間に距離が入り,変形を考える対象になることを理解した.特に重み付き射影直線の導来圏における Serre 不変性および Gepner 型安定性条件の意味を整理し,その構成を確認した.さらに,ADE型箙の表現の圏の導来圏では,複素数による自然な作用を除けば,Gepner 型安定性条件が本質的に一意に定まることを確かめた.\\ \par

一方で,三角圏の半直交分解が与えられたとき,全体の圏の安定性条件と各成分の部分圏の安定性条件の間の一般的な対応はまだ理解できていない.特に,各部分圏上の安定性条件から全体の圏上の安定性条件がどのように構成されるかは明確ではない.現在の理解では,部分圏で定義された安定性条件を適切に整合させることで,全体の圏の安定性条件が張り合わせにより得られるのではないかと期待されるが,その具体的条件や必要十分条件は分かっていない.そのため,まず変異によって移り合う部分圏に対応する安定性条件を比較し,それらを同時に含むような安定性条件を構成することを考えている.\par
この問題が難しい理由は,安定性条件が局所的な圏構造からは決まらない点にある.半直交分解は射の消滅条件を与えるが,安定性条件は拡大構造を用いて定義されるため,各部分圏の核から全体の核を単純に再構成することが出来ない.特に半安定性は部分対象の位相の比較により決まるが,対象の分解は部分圏をまたいで現れるため,部分圏ごとの安定性条件を与えても全体での Harder–Narasimhan 分解が保たれるとは限らない.このため安定性条件は単純な貼り合わせとして構成できず,その整合条件を記述する必要がある.\par
その手始めとして,重み付き射影直線の導来圏と,その半直交分解の成分として現れる ADE 型箙の表現の圏の導来圏においてこの問題を具体的に調べる.特にミラー対称性から期待される構造を手がかりとして,全体の圏と部分圏の安定性条件の関係を同定することを目標とする.\\ \par
安定性条件が構成されると,三角圏の対象が Harder–Narasimhan 分解を持ち,圏を分解可能な形で扱えるようになる.さらに安定性条件の空間は連続的に変形できる幾何的対象となる.本研究では,安定性条件の空間を理解するためには,半直交分解の取り替えに伴って安定性条件がどのように移り変わるかを記述する必要がある.半直交分解を変異によって取り替えると,対応する安定性条件も連続的に変形されるが,その過程で安定対象の集合が変化する.このような移り変わりを安定性条件空間とある配置空間との間の写像を通して理解することを目標とする.とくに,被覆写像の構造を手がかりとして,半直交分解の変化が安定性条件空間にどのような分割を与えるかを明らかにしたい.\end{document}
