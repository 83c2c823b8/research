\section{Bridgeland安定性条件の空間}
\begin{defn}\cite{Bri07}
$\mathrm{Slice(\D)}$を$\D$上のスライスの集合,$\mathcal P,\mathcal Q\in\mathrm{Slice} \D$に対して,
\[d(\mathcal P,\mathcal Q)\coloneq\sup_{0\neq E\in\D}\{|\phi^-_{\P}(E)-\phi_{\Q}^-(E)|,|\phi^+_{\P}(E)-\phi_{\Q}^+(E)|\}.\]
と定義する.$(\mathrm{Slice(\D)},d)$は,距離空間である.
\end{defn}
\begin{lemm}
	$\phi_{*}^+(E),\phi_{*}^-(E)\colon \mathrm{Slice(\D)}\rightarrow\RR$
	は連続写像である.
\end{lemm}
\begin{proof}
	$d(\P,\Q)=0$とすると,$0$でない$\P(\phi)$の対象は$\Q(\phi)$の対象である.したがって,$\P=\Q$.対称性,三角不等式は,絶対値と$\sup$の定義よりしたがう.
\end{proof}

ある有限生成自由アーベル群 $\Gamma$ とGrothendieck群 $K(\D)$ からの群準同型
\[
v \colon K(\D) \to \Gamma .
\]
をひとつとり,固定する.
\begin{defn}[台条件(support property)]\cite{Bri07}
三角圏 $\D$ 上の安定性条件 $\sigma = (Z, \mathcal{P})$ が次の条件が成立するとき,台条件を満たすという:\\
$C > 0$ とノルム\ $\|\cdot\|\colon\Gamma \otimes_{\mathbb{Z}} \mathbb{R} $ が上に存在して,任意の $\sigma$-安定対象 $E \in \D$ に対して
\[
\|v(E)\| \leq C \cdot |Z(v(E))| .
\]
が成立することである.
\end{defn}

\begin{prop}\cite{Bri07}
$K(\D)$ から有限自由アーベル群 $\Gamma$ への固定された群準同型$v \colon K(\D) \to \Gamma$
に対して,$\Stab_\Gamma(\D)$ は次の集合である:
\[
\Stab_\Gamma(\D) \subset \Hom_{\ZZ}(\Gamma, \CC)\times\Slice(\D).
\]
安定性条件の組$(Z,\P)$で台条件をみたすものである.この集合は,忘却写像:
		\[
			\begin{array}{ccccc}
				\mathrm{Stab}_{\Gamma}(\D)& \longrightarrow & \Hom_{\ZZ}(\Gamma,\CC) \\
				\rotatebox{90}{\in}& &\rotatebox{90}{\in}\\
				(Z,\P) & \longmapsto &  Z& .
					\end{array}
\]
は局所同相写像である.とくに,$\Stab_{\Gamma}(\D)$に$\Hom_{\ZZ}(\Gamma,\CC)$の標準的な複素構造に関して,上記の忘却写像が正則写像になるような複素構造が一意に定まる.
\end{prop}
\begin{proof}
	この写像が局所的に単射であることを示す.$\sigma = (Z,\P),\ \tau = (Z,\Q)\in \Stab_{\Gamma}(\D)$の2点に対し,$d(\P,\Q)<1$を満たすなら$\P = \Q$となることを背理法で示す.\\
	$\P\neq \Q$と仮定する.このとき,ある$\phi\in\RR$と$E\in\P(\phi)$が存在して,$E\notin\Q(\phi)$となる.このとき,$Z(E)\in\RR_{>0}e^{i\pi\phi}$である.\\
	$E\in\Q(\ge\phi)$とすると,$d(\P,\Q)<1$より,$E\in\Q([\phi,\phi +1))$となり,$E\notin Q(\phi)$だったので,
	$Z(E)\in\RR_{>0} e^{i\pi\phi}$に矛盾.同様に$E\notin \Q(\P)$も言えることから,
	\[A\rightarrow E\rightarrow B\rightarrow A[1].\]
	で$A\in\Q((\phi,\phi+1))\backslash \{0\},B\in\Q((\phi-1,\phi))\backslash \{0\}$となるものが存在する.
	$d(\P,\Q)<1$より,$A\in\P((\phi-1,\phi +2))$となる.\\
	$A\in\P((\phi -1,\phi])$であるなら,$\sigma$と$\tau$が同じ中心電荷を持つことになり矛盾.したがって,$\psi >\phi$と$C\in\P(\psi)$と$0$ではない分解の射が存在する.
\end{proof}
